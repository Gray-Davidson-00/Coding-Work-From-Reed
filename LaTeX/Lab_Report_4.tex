\documentclass[aps,pre,twocolumn,nofootinbib]{revtex4}

\usepackage{amsmath,amssymb,amsfonts,amsthm}
\usepackage{graphicx}
\usepackage{bbm}


\begin{document}

\title{A measurement of the Electron Charge to Mass Ratio}

\author{Gray Davidson}
\author{Lab Partner: Mary Solbrig }
\affiliation{Department of Physics, Reed College, Portland, Oregon,  97202, USA}

\date{\today}

\begin{abstract}  
The present experiment used a cathode ray tube to fire an electron stream through a magnetic field created by a pair of Helmholtz coils.  The field bent the electron beam in proportion to the strength of the field and the velocity of the beam.  The radius of the resulting deflection was used to measure the charge to mass ratio of the electrons.  The subsequently calculated accepted value, derived from the official charge and mass of the electron was found to lie mare than the range of calculated experimental error away from the experimentally achieved value.  Since the statistical power of the experiment was decently large, it was concluded that this discrepancy was either due to some systemic error, or to an underestimate of the experimental error in one or more quantities.  
\end{abstract}
\maketitle

\section{Introduction}
After Milikan's famous experiment to measure the charge of the electron \cite{Scribner2008}, the further question of its mass remained.  This question was to be answered in the current case with a device consisting of a beam of electrons and a magnetic field B of calculable magnitude, which will deflect the electrons in a measurable way, and thereby imply the ratio of their mass to their charge.  This relationship is given by: 

\begin{equation}
\label{qoverm}
\frac{q}{m}=\frac{2V}{(rB)^2}
\end{equation}

Where r is the radius of the arc the electrons describe when affected by the magnetic field,  q and m refer to the charge and mass of an electron, V is the input voltage to the cathode ray tube, and B is the magnitude of the magnetic field generated by two helmholtz coils.  The equation for the magnetic field generated by a single loop of wire driven by a current I is \cite{Giancoli}:

\begin{equation}
\label{bfield}
B=\frac{\mu_0 I R^2}{2(d^2-R^2)^{\frac{3}{2}}}
\end{equation}

Where R is the radius of the wire loop, b is the distance along the central axis of the loop at which the measurement is taken, and I is the current through the loop.  In the current case, two coils were used, each containing 320 loops (N=320), and they were placed 2R apart, with the beam of electrons exactly between them.  This meant that d=R, and the equation became: 

\begin{equation}
\label{bfieldmodified}
B=\frac{8 \mu_0 I N}{5 \sqrt{5} R}
\end{equation}

The only further quantity of interest is r, the radius f the deflection circle.  This quantity is geometrically calculated by knowing two quantities which are both easier to measure in a laboratory.  These are a, and d, as shown in the following diagram: 

\begin{figure}[h]
\centering
\includegraphics*[width=  .9 \columnwidth]{/Users/Requiem/Documents/Physics 200 - Spring/sector.jpg} 
\caption{A sector of a circle, showing the relationship of the measured quantities a and d to the desired quantity, r.  Their algebraic relationship is displayed in Eq.~(\ref{r}).  }
\label{graph}
\end{figure}

And r is calculated as follows: 

\begin{equation}
\label{r}
r=\frac{a^2+d^2}{2d}
\end{equation}


\section{Methods and Apparatus}
The beam was generated using a cathode ray tube, whose input voltage (V) was controlled by a high-voltage source in the range of 2-6 kV.  The beam was expressed on a sheet that had been marked at with a millimeter grid.  In this way, measurements of a and d were taken with good precision.   Figure~\ref{ratus} demonstrates the apparatus used.  

\begin{figure}[h]
\centering
\includegraphics*[width=  .9 \columnwidth]{/Users/Requiem/Documents/Physics 200 - Spring/ratus4.jpg} 
\caption{Experimental apparatus, showing the measurement of a and d, as well as electrical wiring and the relative positions of the cathode ray tube and the helmholtz coils, which are seen end-on in this diagram.}
\label{ratus}
\end{figure}

The apparatus was assembled as in the diagram, with especial care that the leads from the high voltage source were safe, since a shock from this source could be harmful to humans.  The voltage to the cathode ray tube and the current through the Helmholtz coils were varied across their full range of values for which a stable, readable curve was produced.  The method of calculating q/m is set down in the introduction, and a discussion of the error analysis is included in the appendix section.  

\section{Results}
The experiment yielded a value for the ratio of q/m=1.68 +/- .04*$10^{11}$.  The accepted value, (calculated with the accepted values of q and m is 1.76*$10^{11}$, which lies outside the range of experimental error.  30 measurements were taken in each of two test subgroups, these groups were differentiated by a reversal of the B-field's direction, north or south.  This change was important because one of these conditions allowed the b-field to be the sum of B from the coils and B of the earth, and the other condition allowed the B total to be the difference between these two.  Indeed, the results were different, with the sum condition yielding a higher value.  The average of these two results, weighted for their respective uncertainties was used to calculate the final reported answer 1.68.  The radius of one of the Helmholtz coils was  measured as 6.8 cm, with an uncertainty of .1 cm.  This was consistent for both coils, and also for the distance from each coil to the electron beam.  The reported uncertainties in the other measured quantities: V, I, a, and d respectively were .3kV, 1mA, .5mm, .5mm.   A complete data table is included in the appendix, since there were more data collected than can easily be presented here.  Figure~\ref{data4} displays the final values of q/m.  

\begin{figure}[h]
\centering
\includegraphics*[width=  .9 \columnwidth]{/Users/Requiem/Documents/Physics 200 - Spring/data4.png} 
\caption{This graph demonstrates that the data were quite good.  The agreement among almost all trials is visibly consistent. }
\label{data4}
\end{figure}

It is clear from the graph that the magnetic field of the earth did affect the outcome.  The first 30 trials are clearly higher on average than the following 30.  We have no specific way of explaining the visible outliers, except for the suggestion that these were trials near the extremes of either voltage or current, and thus were affected by the difficulty of taking data in these regions where, for instance, the beam was very faint.  

\section{Discussion}
It is unlikely that this experiment, however carefully carried out, is a sufficiently convincing measure for refuting the accepted value of the electron's mass.  More likely, is the assumption that some errors are present in either the measurement or calculation.  The similarity of the number calculated experimentally to the accepted value indicates that any error in the experiment or analysis is either a small systemic error or is a matter of underestimation in terms of the experimental uncertainties.  This last is by far the most likely problem.  Some concern also springs from the equipment used.  The orientation of the electron beam to the magnetic field was not a carefully calibrated quantity, although all calculations hereto are predicated on the assumption that this was a right angle.  Indeed, even lacking a perfect angle, the reproduction of the experiment in the opposite direction should have removed the effects.  This is mostly true, except that the strength of the magnetic field would have been reduced by a factor of $cos(\theta)$ where $\theta$ is the angle.  Clearly in an ideal case, $\theta=\frac{\pi}{2}$ and $cos(\theta)=1.$  Finally, data points were included in this calculation near the extremes of high current and low voltage, at which values, the beam became very faint on the metric grid, as well as unstable if the voltage were changed any more.  These data points seemed to fit with the overall pattern, so they were included in the calculations, but they were perhaps less trustworthy than data taken in the stable range of 3-6 kV.  

\section{Conclusion}
Although the experimental figure q/m=1.68 +/- .04*$10^{11}$ did not agree with the accepted value  1.76*$10^{11}$ exactly, and fell outside the range of experimental uncertainty according to the estimates made in the laboratory, the similarity of these values suggests that the experiment should be repeated with special attention to the sizes of experimental uncertainties, and to careful measurement, rather than that the experiment is in some way a bad procedure for measurement of this quantity, or that there was some error present in the associated mathematics.  

\appendix*
\section{Error Propagation}
From \cite{Taylor1997}, we lift the following equations: 


 For error propagation through multiplication,

\begin{equation}
\label{multi}
z=\frac{ab}{c}
\end{equation}

the error is:

\begin{equation}
\label{multierror}
\frac{\Delta z}{z}=\sqrt{{\frac{\Delta a}{a}}^2+{\frac{\Delta b}{b}}^2+{\frac{\Delta c}{c}}^2}
\end{equation}

For error propagation through addition, the error is: 

\begin{equation}
\label{adderror}
\sigma (a+b)=\sqrt{(\sigma a)^2+(\sigma b)^2}
\end{equation}

And for the averages used, the simple formula: 

\begin{equation}
\label{mean}
\sigma_{mean}=\frac{\sigma}{\sqrt{N}}
\end{equation}

was insufficient since the uncertainties in various values were different.  In this case, a weighted average was used, where the average value is:

\begin{equation}
\label{wave}
x_{weighted average}=\frac{\sum w_i x_i}{\sum w_i}
\end{equation}

And $w_i$ is given by the reciprocal squares of the corresponding uncertainties.  

\begin{equation}
\label{wsubi}
w_i=\frac{1}{\sigma_i^2}
\end{equation}

And the uncertainty in this new average is 

\begin{equation}
\label{waveerror}
\sigma_{wav}=\frac{1}{\sqrt{\sum w_i}}
\end{equation}

Thus the weighted average takes into account the relevance of each data point, instead of treating each one equally.  

	\begin{thebibliography}{99}
\bibitem{Scribner2008} C. Scribner's Sons, \textit{Complete Dictionary of Scientific Biography Vol. 2},
  (Electronic Edition, 2008).
\bibitem{Taylor1997} J. R. Taylor An Introduction to Error Analysis; The Study of Uncertainties in Physical Measurements, (University Science books, Sausalito, CA, 1997)
\bibitem{Giancoli} Douglas C. Giancoli Physics for Scientists and Engineers, Third Edition, (Prentice Hall, Upper Saddle River, NJ, 2000)
\bibitem{constant} CODATA Recommended Values of Physical Constants: 2006 P. J. Mohr, B. N. Taylor, and D. B. Newell, National institute of Standards and Technlogy (2007).
	\end{thebibliography}
	
	
\end{document}

