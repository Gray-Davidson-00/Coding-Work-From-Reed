%
% As physicist we often submit to physical review journals which have
% the "revtex" environment as default. I therefore use revtex here.
% revtex is however very similar to the standard "article" documentclass environment so
% that there should be no compatibility issues.
%
% There are many options you can choose (in [  ] brackets) - see the revtex manual from
% the american physics society
%
\documentclass[aps,pre,twocolumn,nofootinbib]{revtex4}
%
% some useful packages
%
\usepackage{amsmath,amssymb,amsfonts,amsthm}
\usepackage{graphicx}
\usepackage{bbm}
\usepackage{pdfsync}


\begin{document}

%Title of paper
\title{title}


% Place the author information here.  
% \affiliation command applies to all authors since the last
% \affiliation command. The \affiliation command should follow the other information
% \affiliation can be followed by \email, \homepage, \thanks as well.

\author{You}
\author{Lab Partner: }
\affiliation{Department of Physics, Reed College, Portland, Oregon,  97202, USA}

\date{\today}

\begin{abstract}  
This document provides a template for your report. It is  based on the two-column revtex documentclass.
\end{abstract}


%\maketitle must follow title, authors, abstract, \pacs, and \keywords
\maketitle



\section{Introduction with citations}

Here is how you can insert an article citation \cite{Einstein1905} and a book citation \cite{Arfken2001}. You can also combine them \cite{Einstein1905,Arfken2001}.

\section{some typesetting}

\subsection{Figures}

You may include figures and refer to them from the text. Here we have  two figures, Fig.~\ref{fig:blank} and Fig.~\ref{fig:longblank}. \LaTeX will place floating environments such as the figures according to its own rules. Therefore, to avoid ambiguity, you need to refer to them by their label.
%%%%%%%%%%%%%%%%%%%%%%%%%%%%%%%%%%%%%%%%%%%%%%%%%%%%
% FIGURE 1
%%%%%%%%%%%%%%%%%%%%%%%%%%%%%%%%%%%%%%%%%%%%%%%%%%%%
%% THIS DEFINES A FIGURE.  THE OPTIONAL [h] DECLARATION ASKS TEX TO PLACE IT AS
% CLOSE TO WHERE I TYPED THESE LINES AS POSSIBLE
%
\begin{figure}[h]
\centering
% THE FIRST LINE INPUTS THE GRAPHICS.  THE WIDTH COMMAND IS OPTIONAL AND
% CAN BE OMITTED
\includegraphics*[width=  .9 \columnwidth]{blank} 
% THE NEXT LINE DEFINES THE CAPTION
\caption{A blank figure }
% FINALLY WE LABEL THIS FIGURE SO THAT WE CAN REFERENCE IT AS DONE ABOVE
\label{fig:blank}
\end{figure}

%%%%%%%%%%%%%%%%%%%%%%%%%%%%%%%%%%%%%%%%%%%%%%%%%%%%


%%%%%%%%%%%%%%%%%%%%%%%%%%%%%%%%%%%%%%%%%%%%%%%%%%%%
% FIGURE 2
%%%%%%%%%%%%%%%%%%%%%%%%%%%%%%%%%%%%%%%%%%%%%%%%%%%%
%
%There are alternative versions of the float environments - in two-column documents, figure* provides a floating page-wide figure (and table* a page-wide table). The "*"ed float environments can only appear at the top of a page, or on a whole page - h or b float placement directives are simply ignored.
% 
\begin{figure*}[hbtp]
\centering
\includegraphics*[width= 1.6 \columnwidth]{longblank} 
\caption{A blank figure better viewed spanning two columns}
\label{fig:longblank}
\end{figure*}
%%%%%%%%%%%%%%%%%%%%%%%%%%%%%%%%%%%%%%%%%%%%%%%%%%%%

\subsection{Equations}

\begin{equation}
\label{2and2is4}
2 + 2 = 4
\end{equation}

\begin{align*} % align, aligns equations vertically at tthe &.  * if you want un-numbered equations
\text{erf}(z) &= \frac{2}{\sqrt{\pi}} \int_0^z e^{-t^2} dt    \\
 \text{erf}(z)& =  \frac{2}{\sqrt{\pi}} \sum_{i=1}^{\infty} \frac{(-1)^k \, z^{2 k + 1}}{k! \,(2 k + 1)}
\end{align*}

\begin{equation}
\begin{split} % puts one  number next to whole construct
F(v) &= V_0 - \sqrt{ F_1(v)^2 + a^2} \qquad \text{with} \\
F_1(v) &= 
\begin{cases}
A_l \; (v - v^*) &  \text{  if }\; v \le v^* ,\\
A_r \; (v - v^*) & \text{  if  }\; v > v^* .
\end{cases}
\end{split}
\label{tentnl}
\end{equation}


\begin{equation}
\label{matrixeq}
\frac{d}{ds} 
\begin{pmatrix}
          x \\
          y 
\end{pmatrix}
=
\begin{pmatrix}
          -1 & 1 \\
          -r & 0 
\end{pmatrix}
\begin{pmatrix}
          x \\
          y 
\end{pmatrix}
- \gamma \; \mathbbm{1} \;
\begin{pmatrix}
 f\left[ x(s - \hat{\tau}) \right] \\
0
\end{pmatrix}
\end{equation}
 
 
 You refer to these equations by their label, e.g. Eq.~(\ref{2and2is4}), Eq.~(\ref{matrixeq}), and Eq.~(\ref{tentnl}).

\subsection{Tables}
%
%
% HERE WE INCLUDE A TABLE.  THE [h] COMMAND HAS THE SAME EFFECT AS IN THE
% FIGURE ENVIRONMENT ABOVE
\begin{table}[h]
	% THE CAPTION COMES FIRST FOR A TABLE
	\caption{Experimental data.  Description goes here }
	% NEXT A RULEDTABULAR WILL PLACE DOUBLE LINES ABOVE AND BELOW THE TABLE
\begin{ruledtabular}
	% THIS COMMAND ACTUALLY BEGINS THE TABLE.  THE COMMAND {ccccc} TELLS LATEX 
	% THAT THIS TABLE HAS FIVE CENTERED COLUMNS.  THE COMMAND {crl} WOULD 
	% PRODUCE THREE COLUMNS.  THE FIRST CENTERED, THE SECOND RIGHT-JUSTIFIED,
	% AND THE THIRD LEFT-JUSTIFIED
	\begin{tabular}{ccccc} 
	% FOR EACH ROW, THE ELEMENTS ARE SEPARATED BY &.  THE COMMAND \\ ENDS A 
	% ROW AND A \hline COMMAND CAN BE USED TO DRAW A HORIZONTAL LINE SEPARATING 
	% THIS ROW FROM THE NEXT
	$R \, (\mathrm{k} \Omega)$ & $I_c \, (\mathrm{m} A)$ & $V_c \, (V)$ & $I_b \, (\mathrm{m} A)$ & $\beta$ \\  \hline
	0 & 11.67 & 0.04 &  0.94 &  12.4 \\ 
	47 & 11.54 & 0.17  & 0.09  & 130 \\ 
	100 & 8.33 & 3.3  & 0.04  & 189 \\ 
	470 & 1.90 & 9.78  & 0.01  & 196 \\ 
	1000 & 0.9 & 10.75  & 0.005  & 196 \\ 
	\end{tabular}
	\end{ruledtabular}
	% AGAIN A LABEL IS PROVIDED SO THAT THIS TABLE CAN BE REFERENCED
	\label{tab:data}
\end{table}

 You refer to the table by its label, e.g. Table~\ref{tab:data}.




%\begin{acknowledgments}
%you might want to thank someone here - but this is optional
% \end{acknowledgments}


\appendix*
\section{}

The appendix is for content that would detract from the main arguments in the paper but is important for completeness, e.g. lengthy details of your error analysis could go here. 

% The bibliography goes here. It is standard to order the list such that the article or book cited first in the text is first on the list.   
%
 	\begin{thebibliography}{99}
%
\bibitem{Einstein1905} A. Einstein ``Zur Elektrodynamik bewegter K\"orper,"  \textit{Annalen der Physik} \textbf{17}, 891 (1905).

\bibitem{Arfken2001} G. B. Arfken and H. J. Weber, \textit{Mathematical methods for physicists},
  (Harcourt/Academic Press, San Diego, 2001).

%
	\end{thebibliography}



\end{document}

















