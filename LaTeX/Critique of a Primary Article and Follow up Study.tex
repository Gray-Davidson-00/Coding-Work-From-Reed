\documentclass[aps,pre,twocolumn,nofootinbib]{revtex4}

\usepackage{amsmath,amssymb,amsfonts,amsthm}
\usepackage{graphicx}
\usepackage{bbm}

\renewcommand{\baselinestretch}{2}

\begin{document}

\title{Critique of a Primary Article and Follow up Study}

\author{Gray Davidson}
\author{Professor: Paul Currie }
\affiliation{Department of Psychology, Reed College, Portland, Oregon,  97202, USA}

\date{\today}

\renewcommand{\baselinestretch}{1}

\begin{abstract}
This examination attempts to summarize the relevant background information on the subjects of creativity and mood disorders.  Secondly, it scrutinizes a study published by researchers at Stanford University, conducted in 2006.  And finally, it proposes a direction for further research.  
\end{abstract}
\maketitle


\section{Background}

The archetype of the mad genius is not a new idea in western culture, but detailed analysis of that archetype is comparatively recent.  For the past three decades tests of the correlation between bipolar disorder (BD), Major Depressive Disorder (MDD) and creativity have been common, and have been long-term works for several psychologists (Jamison et. al. 1995, Andreasen).  These tests show only half the story, however, in that they commence their exploration of the subject by selecting �eminently creative individuals,� \ref{Crocq} \ref{Figuerosa},  such as Andy Warhol, or Virginia Woolf using these individuals� eminence as a measure of creativity.  These studies do indeed support the correlation between mood disorders and enhanced creativity, but their failing is, among other things, their inapplicability to non-eminent individuals already diagnosed with mood disorders, but not situated in spotlights of the creative industry.   
In addition, there is a heavy reliance on creativity as it is found in writers and visual artists.  This reliance excludes persons with other varieties of creativity-problem solving and diplomacy, for example, or in the disciplines of pure math or theoretical physics.  To be fair, this sort of creativity is difficult to measure, or even to define.  One study, published in 2008, claimed SAT-Math scores in adolescence were a good predictor of future �creativity,� as measured by the level of degree (BS, MS or PhD) earned, and number of patents or publications (Park et. al. 2008).  Clearly some liberties have been taken with the definition of creativity.  
A neural approach may be a better strategy in classifying creativity, or at least in defining differences in types of creativity.  A good place to start is with types of mental imagery.  Artists have been found to exhibit object imagery, characterized by holistic views of an object, whereas scientists, including engineers, have been found to exhibit spatial imagery�examining mental objects piece by piece (Koshevnikov et. al. 2005).  One question is whether this difference is a cause or a result of the career choices made by these individuals.  
The salient question, however, is whether this difference in mental imagery is translatable to a measurable difference relative to creative controls or relative to mood disorder patients in the results of creativity tests.  The mad scientist is as much a western archetype as the mad genius, but the connection between these is an unexplored field.  
In addition to the past failings in terms of content, it is rare that any study of creativity, as it connects to mood disorder has taken the obvious step of examining the neural or chemical localization of the peculiarity, to look for a connection in terms of the neurotransmitters involved.  One study used near-infrared optical spectroscopy (NIRS) methods to assess the cerebral locality of activation due to a �creative� task, and found bilateral prefrontal activation relative to controls.  In addition, the same procedure was applied to schizophrenics and non-schizophrenic schitzotypes.  Between schitzophrenics and healthy controls, no difference was observed, meaning that bi-lateral activation still occurred in the prefrontal cortex, but in schitzotypes, activation was statistically higher in the right prefrontal cortex, a finding which correlated with higher creativity in those subjects \ref{Folley}!  The partial localization of creative function to the prefrontal cortex is confirmed by other inspections as well\ref{dietrich}, and several sources agree that this function is at least to some extent lateralized to the right hemisphere \ref{dietrich} \ref{Folley}.  
	There exists, of course, a significant problem in the classification of creativity, although the article whose critique is to follow does not address this issue, preferring to leave the question to those who developed the tests they used.  One important distinction is between a conscious process and an unconscious.  To fall back to anecdotal evidence, stories of Samuel Taylor Colridge's "Kubla Kahn" or Singer's sewing machine (which both appeared fully formed in their creators' dreams) would seem to imply that the latter is at least an element of the total.  The decision is not important when studying mood disorders, however, since the several tests included in the present study are designed to show a significant difference between creative individuals and controls, and can therefore be used to test whether mood disorder patients display the same characteristics as creators.  
	





\section{Critique}

A recent set of reports from Stanford University�s department of psychiatry and behavioral sciences sheds light on the issue from a new direction.  The report, written by Santosa CA, Strong CM, Nowakowska C, et al. selects populations of patients with BD, MDD, a population of healthy controls, and a population of creative controls (defined in this case by profession: architect, painter, poet, etc.).  Each of these groups contained on the order of 30 individuals, and although the authors claimed lack of participants �limited [their] statistical power,� (Santosa CA et al. 2006) they have no reason to make this claim except in several limited cases.  For instance, the participants did not differ statistically in any way except that creative controls were younger and had two years more of education.  To counter this, the authors limited for some analyses their examination to subjects under 40 years old, which  eliminated the discrepancy.  Unfortunately this limitation also apparently reduced some test groups to the point that the resulting data cannot be trusted explicitly.  At other times, these four groups were divided on the basis that some participants were receiving certain kinds of treatments for their mood disorders, or had clinical history of mood disorders.  
The four groups of participants were administered a battery of tests, to determine temperament, creativity and personality.  These tests consisted of, in the case of creativity, the Barron Welsh Art Scale (BWAS), the Adjective Checklist Creative Personality Scale, and the Torrence Test of Creative Thinking Verbal and Figural (TTCT-V and TTCT-F).  And in the case of personality and temperament, the tests were The Revised NEO Personality Inventory, (NEO-PI-R), the Temperament and Character Inventory (TCI), and the Temperament Evaluation of the Memphis, Pisa, Paris and San Diego autoquestionaire, (TEMPS-A).  




	\begin{thebibliography}{99}
\bibitem{Crocq} Crocq, M. A-, 2007.  Clinical Case.  Bipolar Trouble and Literary Creativity: The Russian Writer Nicolas Gogol.  Encephale-Revue de Psychiatrie Clinique Biologique et Therapeutique 33, 672-676.  
\bibitem{Figuerosa} Figuerosa, G. 2005. Virginia Woolf as an example of a mental disorder and artistic creativity.  Revista Medica de Chile, 133-11, 1381-1388.  
\bibitem{Simeonova} Simeonova DI, Chang KD, Strong C, and Ketter TA.  Creativity in Familial Bipolar Disorder.  Journal of Psychiatric Research 39-6, 623-631.  
\bibitem{Park} Park G, Lubinski D, Benbow CP, 2008. Ability differences among people who have commensurate degrees matter for scientific creativity.  Psychological Science 19(10).  
\bibitem{Kozhevnikov} Kozhevnikov M. Kosslyn S. and Shephard J. 2005. Spatial versus object visualizers: 
A new characterization of visual cognitive style.  Memory and Cognition 33(4).
\bibitem{Folley} Folley BS, Park S, 2005. Verbal creativity and schizotypal personality in relation to prefrontal hemispheric laterality: a behavioral and near-infrared optical imaging study. Schitzophrenia Research 80(2-3).  
\bibitem{dietrich} Arne Dietrich, 2004.  The Cognitive Neuroscience of Creativity.  Psychonomic Bulletin and Review 11(6).  
	\end{thebibliography}
	
	
\end{document}


