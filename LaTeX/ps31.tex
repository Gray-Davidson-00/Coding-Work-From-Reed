\documentclass[aps,pre,nofootinbib]{revtex4}

\usepackage{amsmath,amssymb,amsfonts,amsthm}
\usepackage{graphicx}
\usepackage{bbm}
\usepackage{pdfsync}
\usepackage{color}
\usepackage{hieroglf}

\begin{document}

\title{Problem Set XXXI}

\author{Gray Davidson}
\affiliation{Department of Physics, Reed College, Portland, Oregon,  97202, USA}
\date{\today}
\maketitle

\section{31.1}
a)

From the boundary conditions, we know the following about the electric fields: 

$$\vec{E_I} + \vec{E_R} = \vec{E_{\alpha}} - \vec{E_{\beta}} ~~\mbox{and}~~ \vec{E_{\alpha}} + \vec{E_{\beta}} = \vec{E_T}$$

%\begin{eqnarray}
% y &=& x^4 + 4      \nonumber \\
%   &=& (x^2+2)^2 -4x^2 \nonumber \\
%   &\le&(x^2+2)^2
%\end{eqnarray}

Likewise for $\vec{B}$, we know that 

$$\vec{B_I} + \vec{B_R} = \vec{B_{\alpha}} - \vec{B_{\beta}} ~~\mbox{and}~~ \vec{B_{\alpha}} + \vec{B_{\beta}} = \vec{B_T}$$

Bearing in mind our given: $ \mu_1 \cong \mu_2 \cong \mu_3 \cong \mu_0 $
\bigskip
We note that the electric fields point in $\pm \hat{x}$ and the magnetic fields in $\pm \hat{y}$.  

We write out these four equations in terms of their amplitudes.  



\Hq

\end{document}