\documentclass[jou]{apa}

\usepackage{graphicx}
%\usepackage{natbib}
\usepackage{pdfsync}

\title{Longitudinal Effects of Mindfulness Trainings on Defensiveness and Self-esteem}
\author{Gray D. Davidson}
\affiliation{Department of Psychology\\ Reed College}




\abstract{Mindfulness as a cure for existing psychological ailments or as a beneficial practice for healthy individuals has received considerable press and study in the past decades.  Mindfulness is a practice and theory borrowed from Buddhism, which emphasizes attention in the present moment, and an absence of judgement about what one sees \cite{keng2011}.  Self-esteem has been a controversial topic in the latter half of the $21^{st}$ century, with the debate ranging from whether it exists to whether higher self-esteem is better than lower, and finally to what the field of psychology should recommend as a path toward improvement.  This article reports the effects of individual differences and training in mindfulness on the defensive behavior of persons with high self-esteem in a longitudinal test.  Findings indicated that persons with high mindfulness scores are less likely to be defensive, and that mindfulness training can reduce defensiveness in those with low trait mindfulness.  Further results, from tests administered six months later show that for this benefit to be lasting, a single session is insufficient, and a full course in mindfulness training is necessary.}
\acknowledgements{Gray Davidson, Department of Psychology, Reed College, Portland, Oregon.

This research was supported by a grant from the Social Self Alternative Methods Institute.  Thanks to Kathryn Oleson, Ph.D. Miranda Sitney, Adrienne Wise, and Salvador Rodriguez for their insightful comments on drafts of this article.

Correspondence concerning this article should be addressed to Gray Davidson, Reed College box 272, 3203 SE Woodstock Blvd. Portland, OR.  97202.  E-mail: davidsog@reed.edu}
\rightheader{G. D. Davidson / J. Personality \& Soc. Psych. 101 (2011) 526 - 531}
\shorttitle{Short title for manuscript header}
\leftheader{G. D. Davidson / J. Personality \& Soc. Psych. 101 (2011) 526 - 531}



\journal{Journal of Personality and Social Psychology}
\volume{2011, Vol. 101, No. 6, 526 - 531}
\ccoppy{Copyright 2011 by the American Psychological Association}
\copnum{0022-3513/07/\$12.00 \quad DOI: 10.1037/0022-3513.92.3.527}



\begin{document}
\maketitle

Mindfulness is a practice, secularized from Buddhism which involves ``awareness and nonjudgmental acceptance of one's moment to moment experience," \cite{keng2011}.  While it may seem counter-intuitive that such a small change in cognitive assignment could have far-reaching consequences, mindful individuals, and those who undergo mindfulness training experience widespread improvements in their lives, in terms of mental health, cognitive abilities, and social experience.  It is not obvious how mindfulness should affect self-esteem and responses to threats against the self, nor that it should differentially affect people depending on their individual types of self-esteem, but prior research has hinted that this may indeed be the case.  

What has been conspicuously ignored in the literature is any examination of the long-term effects of mindfulness on the social self.  It is this oversight which I attempt to correct with an initial study on the effects of mindfulness on verbal defensiveness. Lakey, Kernis, Heppner \& Lance examined this relationship in the short term, but neither did their study use any manipulation of mindfulness, nor did they test the longevity of their results \cite{lakey2008}.  These deficits are corrected in the present investigation.  Two separate questions are asked which are natural responses to Lakey et. al.'s research.  First, I ask whether mindfulness training in a single session or across multiple sessions, can improve (that is decrease) an individual's defensiveness score.  Secondarily, I ask whether any benefits felt from these types of mindfulness training can still be detected after some time has passed.  These questions are extremely important, not only for healthy individuals wishing to improve their lives through mindfulness, but are of surmounting importance for individuals with mental illnesses who may benefit from mindfulness-based therapies such as Mindfulness-Based Stress Reduction (MBSR) or Mindfulness-Based Cognitive Therapy (MBCT).  

\subsection{Mindfulness: Meditation for Health}

Mindfulness is a practice with strong roots in Buddhist tradition, originally consisting of such practices as attention to breath, to bodily sensations, to visual and auditory sensations, attempting to carry each of these out without judgement.  As soon as judgmental thoughts do arise, they are to be let go in favor of `noticing' what is happening.  Mindfulness was not supposed to be a mode of seeing and interacting which could be practiced briefly and then somehow incorporated into a separate life, but was supposed to be the only way of passing through life.  This is not an easy practice, and so the exercises, such as attention to breathing or slow walking and sitting meditation are induction strategies to bring one closer to the essence of the practice.  

The use/practice of mindfulness has the potential to lead to a variety of positive improvements, as empirical research in the later half of the $20^{th}$ century has demonstrated. These areas include, but are not limited to: self-esteem, competence, empathy, optimism, and includes negative correlations to rumination, social anxiety, neuroticism, absent-mindedness, and many others \cite{keng2011, brown2007}.  The very definition of mindfulness makes it a practice which must influence the self.  Furthermore, mindfulness has strong neurological \cite{creswell2007} and physiological \cite{wallace1970} effects.  As will be seen, an individuals' mindfulness scores also correlate strongly with their scores on measures of self-esteem.

In clinical psychology many therapists use secularized mindfulness techniques to help their patients.  Several of these programs are Mindfulness-Based Stress Reduction (MBSR) or Mindfulness-Based Cognitive Therapy (MBCT) which enjoy considerable success in aiding patients in controlling negative thoughts (nonjudgemental acceptance) and avoiding negative thoughts about the future and past (of one's moment to moment experience).  These practices emphasize stepping out of automatic pilot with regard to one's cognitions especially, but also with respect to incoming stimuli, and focus on the inherent assumptions which make negative thoughts into negative influences in one's life.  We must all remember that a thought is not a fact \cite{fennell2004}.

The potential benefits of mindfulness are clearly very powerful, and all the methods and practices seem unified by various core principles, but much research remains to be done to evaluate this wealth of diversity.  

\subsection{Self-Esteem: Dimensions and Debate}
Although all researchers agree on some parts of the definition of self-esteem, other elements are hotly debated, as is the relationship between (the search for) self esteem and health.  Definitions vary widely but share common components.  Self-esteem is ``...a subjectively experienced, affective response to the self, (e.g. \cite{brown1993}).  That is, self-esteem is experienced as a feeling-state, elicited in response to the attitude-object ``the self," and ranging in valence from intensely negative/unpleasant to intensely positive/pleasant," \cite{bosson2006}.  

The simpler definition of self-esteem as the collection of an individual's valanced self-thoughts is fairly well accepted, although each research laboratory rewords it to fit a specific theory of self-esteem.  What is least well defined among researchers is the purpose of self-esteem.  Terror Management Theory (TMT) \cite{greenberg2008}, sociometer theory,  \cite{leary1995} and others \cite{crocker2004} all propose different (and equally arguable purposes for self-esteem.  These motivations vary wildly, and it is not even clear whether self-esteem deserves a place as a fundamental motivation or not \cite{ryan2003} especially since the pursuit of self-esteem can be harmful to the individual \cite{crocker2004}. 

The lack of clear definition or delineation of the involved motivations has led to a cascade of overlapping self-esteem dimensions, some of which are better established than others, and some of which overlap to a considerable degree.  Firstly and most obviously, there is a dimension of valence, as was mentioned in the definition above.  High and Low self-esteem have always been present, and can be measured via questionnaires such as the Rosenberg Self-Esteem Scale (RSES).  One strong argument in this field is over the controversial benefit of seeking high self-esteem.  There is no doubt that the right type of high self-esteem is associated with many benefits, but the actions taken in order to achieve high self esteem can sometimes be damaging to one's overall success, such as ``...challenge negative information about the self; they are preoccupied with themselves at the expense of others; and when success is uncertain, they feel anxious and do things that decrease the probability of success but create excuses for failure, such as self-handicapping or procrastination, \cite{crocker2004a, pyszczynski2004}".  

Self-esteem can be either explicit (the traditional concept) or implicit.  Implicit self-esteem must be non-conscious, and can be measured only indirectly via tasks such as name letter preference or IATs \cite{bosson2006, jordan2006, koole2001}.  Further scales include fragile vs. secure high self-esteem \cite{kernis2010}, defensive vs. secure high self-esteem \cite{jordan2003}, and finally whether self-esteem is even a universal trait, or whether it is a uniquely North American phenomenon \cite{markus2003, heine1999, sedikides2003, heine2005}.  The intercultural debate about self-esteem is important for this investigation because mindfulness is a traditionally Eastern practice which is being applied to the potentially Western concept of self-esteem.  

Of these, defensiveness is particularly important in relation to the current research, as it has been shown to respond to mindfulness manipulations.   Hence, the present study focuses on this dimension of self-esteem. Defensiveness is conceptualized as biased processing of social feedback information, and this can occur either when information is collected selectively, or when it is interpreted in a biased manner \cite{schroder2007}.   An example of the difference between accepting and defensive behavior would be the way that ``some people willingly acknowledge the contribution of inadequate preparation to their poor academic performance, whereas others instead belittle the quality of the exam, class, and instructor, \cite{lakey2008}"  Defensive high self-esteem has been shown to overlap in many ways with another dimension, that of discrepant self-esteem \cite{lambird2006} which is the case of high explicit self-esteem and low implicit self-esteem \cite{jordan2006}.  Defensiveness is one of many defense mechanisms against threats to the self.  Defensiveness is a good choice for study in the present case because it has been shown to respond to mindfulness manipulations \cite{lakey2008}.  


Defensive self-esteem is thought to arise from a discrapancy between explicit and implicit self-esteems, with one of these high and the other low.    Findings in this work support the notion that discrepant self-esteem is maladaptive and causes defensive responses to even ambiguous feedback \cite{schroder2007}.  Further work showed that aggression was a result of defensive-high, but not secure-high self-esteem in children \cite{sandstrom2008}.  Taken together, these results and those above indicate that if mindfulness were to be able to decrease defensiveness, the benefits to an individual's social health could be large.  It is toward this line of inquiry that this study is directed.  


\subsection{Mindfulness Effects on Self-Esteem}

Level of mindfulness correlates moderately ($r=.32$) with global self-esteem \cite{lakey2008}.  Improvements to mindfulness have been shown to decrease defensiveness \cite{lakey2008} and social anxiety \cite{rasmussen2011}, to improve interpersonal wellbeing \cite{cohen2009}, and to decrease aggression and ego-involvement \cite{kernis2010}.  Mindfulness is a way of clearing ones thoughts and therefore avoiding false social cognitions such as unfounded doubts of a romantic partner's devotion.  ``Openness to experiencing what ``is" in the present moment, without defending against it, facilitates integrated functioning, aiding the ability to act congruently with respect to one's perceptions, goals, and values, \cite{ryan2003}."

The strong relationship between mindfulness and the social self should perhaps not be surprising given the integral role of the self in the definition of mindfulness, but a conflicting opinion may be derived from the original Buddhist concept of mindfulness.  From this perspective, ``the self of `self-esteem' is a reification, a constructed image that leads people to be overly attached to achievements, possessions, and relationships despite the true impermanence and interdependent origins of such things, \cite{ryan2003}."  So the ultimate end of mindfulness is an absence of a self.  Without a self to defend, this also eradicates defensiveness in a truly mindful individual.  This is not to be without individuality, but to be without ego \cite{ryan2003}.  This concept of selfless-ness is described in very similar ways to a highly developed and well-adjusted self which also does not display any defensiveness or ego.  

The overarching question then, is whether the quality of mindfulness is improving self-esteem, or is allowing mindful individuals to forego the damaging quest for self-esteem.  


Laying this question aside for the time being, this investigation focuses on broadening the research around defensiveness and self-esteem as these are affected by mindfulness trainings of varying lengths.  In this study, the question to be answered is: Do different intensities of mindfulness training have different effects either immediately or over a longer period of time?  This is to be studied with two trainings, a one-session intensive and a six-week class in mindfulness, and is to be assessed by a measure of defensiveness at three time periods, before training, immediately after training, and six months after training. 

The motivation for studying defensiveness, rather than mindfulness itself as the result of a mindfulness training is that defensiveness is a significantly negative behavior which has been shown to be affected by mindfulness.  If it could be shown that mindfulness decreases this behavior months later, then the implications would be very positive for the individual.  

\section{Methods}
I assessed defensiveness, self-esteem and mindfulness with questionnaires, administered repeatedly several times over the testing period.  A self-threat was administered via a final questionnaire as was done previously \cite{lakey2008}, and mindfulness trainings were either a one-day intensive, 3 hour class or a six-week program, totaling 18 hours overall (two 1.5 hour classes each week).  In prior literature studying mindfulness, as well as the normal courses of MBCT and MBSR, six week programs and single sessions are the most commonly used timeframes.  The research hypothesis was that the same pattern of results would emerge in all conditions, (\textit{i.e.} that mindfulness training would decrease defensiveness scores), but that these results would be more acute in highly defensive participants.  Furthermore, I hypothesized that the positive effects of mindfulness training would be immediately apparent in all manipulation conditions, and only last across a 6 month gap if the participants had taken a longer, more intensive course.  The length of this gap was chosen to be long enough that the effect would have had a chance to fade if it was not to be retained, but not to be so long as to be prohibitive from a research standpoint.  Predicted effects of mindfulness were a significant reduction of defensiveness and elevation of mindfulness in participants, but that these changes would largely disappear again in the single session group, and not in the group which received multiple sessions of training.  


\subsection{Participants}
Participants were high self-esteem college students (N = 140; 55\% female) from the Portland Oregon area, recruited at Reed College, Lewis \& Clark College, Portland State University and Portland Community College.  The average participant age was 26 years old, (SD = 4.88).  Recruitment methods included flyers, advertisements in campus newsletters, and internet social networking methods with the majority of participants responding to campus newsletters.  These advertisements indicated that applicants could receive free training in mindfulness in exchange for a minimal time-commitment participation in a non-aversive research program.  These data were collected over the course of three school semesters with roughly a third of the participants each time.  Each cohort was divided equally into all conditions, and all conditions contained enough participants for statistically significant results.  It is not believed that the time separation between cohorts influenced the results.  
 
\subsection{Measures}
The \textit{Mindful Attention Awareness Scale (MAAS)} was chosen as a measurement of mindfulness because of its common use in prior research.  The scale consists of 15 questions which are answered on a 6-point scale from `almost never' to `almost always.'  These statements are designed to capture the various pieces of the definition of mindfulness, for example ``I rush through activities without being really attentive to them, \cite{brown2003}"   Cronbach's alpha for the MAAS has been consistently above .80, and in this study it was .85, thus adequate internal reliability was demonstrated \cite{lakey2008, brown2003}.  

Self-esteem was evaluated with the \textit{Rosenberg Self-Esteem Scale (RSES)}, which is a 10-item scale with each item rated on a 4-point scale from `strongly agree' to `strongly disagree.'  The items were statements about life satisfaction and liking for the self.  This scale did not access any dimension of self-esteem except explicit, self-reported self-esteem, and made no distinction between fragile, stable, secure and defensive self-esteems.  Cronbach's alpha was .86, and although prior studies report it in the range of .89 \cite{lakey2008}, internal reliability was adequately shown for this investigation.  

Finally, defensiveness was measured via an induction interview, the \textit{Defensive Verbal Behavior Assessment (DVBA)} \cite{barrett2002}.  The interviewer would read a series of open-ended questions to the participant who would respond however they chose.  The first set of questions were designed to introduce the participant to the format and to make them comfortable speaking.  The next set of questions was designed to make the participant defensive by eliciting ego-dystonic responses.  The questions in this section included ``Tell me about a time you did something morally wrong."  Finally, the last few questions were designed to restore the participant to equilibrium.  Interviews were video-recorded and scored later by several extensively trained undergraduates.  Ten interviews were dual coded to ensure inter-rater reliability, which was sufficiently high (ICC = .91).  Each interview took at least a half hour to complete, and the analysis process was time consuming, but as the sample was spread over three cohorts, and each cohort was interviewed at different times, a steady stream of interviews was conducted and this measure did not delay publication time much \cite{barrett2002}.  This was the tool utilized by \cite{rasmussen2011}.


\subsection{Procedure}
The research was conducted in three stages, beginning with an initial screening session where self-esteem was measured.  All respondents were offered a course in mindfulness, but those with high self-esteem were allowed to continue the study. The initial screening was followed by the first of three testing phases $t_0$, in which defensiveness and mindfulness were measured (MAAS and DVBA).  The study up to this point was effectively a replication of prior research, and similar results were expected \cite{lakey2008}.  At this point I expected there to be a correlation between mindfulness and self-esteem, and negative correlations between both of these measures and defensiveness.  Individuals with high self-esteem were selected as participants because the bulk of prior research has concentrated on binary divisions of high self-esteem, (such as fragile s. secure).  The high self-esteem group was then randomly assigned to two groups.  One group received an intensive, 3 hour course in mindfulness meditation which was a dramatic overview of methods and theory, and also included exercises and participation.  The other group received a 12 session course in mindfulness which included the same content, but on a much wider timescale.  The extended course was taught in two 90 minute sessions each week for six weeks.  The single mindfulness training condition received their 3-hour course on the same day as the final session of the extended course, and the second testing phase was administered to all participants in the next few days.  This testing phase served to answer whether the mindfulness trainings caused immediate changes in the defensiveness or mindfulness scores of trainees.  The conclusion of this phase started the clock on the wait period, which lasted six months, and concluded in a third and final test phase in which the same assessments (MAAS and DVBA) were again made.  Comparison of $t_1$ and $t_2$ test scores would allow the lasting effects of mindfulness training to be distinguished from the immediate ones.  Following the final test phase, participants were debriefed.

The design is a 2 (high or low defensiveness) by 2 (high or low mindfulness), tested at three separate times ($t_0$, $t_1$ (+ 1.5 mo.) and $t_2$ (+ 6 mo.)).  Predicted results are a replication of Lakey et. al. 2007 at $t_0$, that the differences between groups would disappear at $t_1$, and that these would reappear at $t_2$ only for the group whose mindfulness training was brief.   Correlations are to be calculated between the three measures at $t_0$, and defensiveness is the dependent variable at later times, after the mindfulness manipulation has been accomplished.  

\section{Results}
Participants were asked at the end of the final testing session, what they thought was the research hypothesis, and none guessed correctly.  12 participants did not complete one or more of the testing sessions, and their data were excluded.  Data from 138 participants was included in the final analysis.  No effects of gender were found on any measures at any testing phases.  

At $t_0$, defensiveness and mindfulness measures divided the sample into four groups, reporting baseline mindfulness and defensiveness scores for each of these These scores were each split into two groups, and are shown in Fig. 1 (a).  As in prior research, a moderate negative correlation (r = -.27) was found between mindfulness and defensiveness scores at this time.  Further correlations (self-esteem \& mindfulness (r = .30) and self-esteem and defensiveness (r = -.38) were also found, and this pattern of results replicated the results of Lakey et. al. 2007.  

At $t_1$ the groups that had previously scored high on defensiveness and low on mindfulness had respectively lowered and raised their scores so as to be nearly indistinguishable from their counterparts with low defensiveness and high mindfulness.  Thus at $t_1$ all cells scored about the same on the test of defensiveness, and the mindfulness manipulation was shown to be successful in improving mindfulness scores in those with low mindfulness.  At $t_2$, those participants who had received six weeks of training appeared no different than they had at $t_1$, indicating that the mindfulness training had a lasting effect.  This comparison is indicated by a solid box in Fig. 1.  By contrast, the effects from a single-session mindfulness training had disappeared, and these participants' scores were not significantly changed from their baseline scores.  This comparison is indicated in a dashed box in Fig. 1.

\begin{figure}[h]
\centering
\includegraphics*[width=  .9 \columnwidth]{results} 
\caption{The results collected in this investigation.  The initial testing $t_0$ is shown at left, and these data are copied to form the bottom row for comparison at later times.  It is assumed that no spontaneous changes in mindfulness and defensiveness occur in a population.  The two mindfulness training programs are indicated by M6 (six week program) and M1 (single session program).  Vertical arrows indicate an improvement from baseline, whereas an O indicated equality with the baseline measurement.}
\label{results}
\end{figure}

\section{Discussion}
This investigation sought to find whether manipulations of mindfulness would firstly have effects on the defensiveness measures in participants, and secondly whether these effects would last across a six month wait period.  Baseline measures of defensiveness showed that the targeted groups were indeed isolated.  Four conditions were studied, all permutations of defensive high self-esteem (high and low defensiveness) with high and low mindfulness, and assessments were made at $t_0$: baseline $t_1$: immediately following mindfulness training, and $t_2$: six months after the end of training.  Results showed that while participants with low defensiveness scores performed well at all time periods (although their results did improve somewhat), those participants with high defensiveness were helped greatly by the mindfulness training.  These results were only lasting however, if the training was a thorough one which lasted over six weeks.  Interestingly the condition which started with high mindfulness and high defensiveness was also helped by the mindfulness training even though their mindfulness scores improve only slightly.  This would seem to indicate the possibility that mindfulness training is improving abilities and thought patterns which are not necessarily captured by the mindfulness measures employed here.  

The primary limitation of this study was the different time lengths of mindfulness training (3 hrs, vs. 18 hrs.).  This difference means that the results obtained can easily be interpreted as the result of either a longer or a more spread out training period or both.  Further research can perhaps tease apart these two elements of what has been shown to be a successful training program.  

A secondary limitation is the use of defensiveness as a dependent variable.  Instead, a set of variables could beneficially be used to show a widespread positive effect of mindfulness.  Defensiveness was chosen in this case because it has been examined in the past, but fragility of self-esteem is a good candidate, as are health effects.  

In the past, it has been shown that even small mindfulness exercises can have large effects and that research is supported by the results of the current study.  A single class in mindfulness was able to lower the scores of those with high defensiveness and raise the scores of those with low mindfulness so that at the $t_1$ test, these were indistinguishable from their counterparts with high mindfulness and low defensiveness.  

Prior research \cite{lakey2007} investigated the relationship of mindfulness, self-esteem and defensiveness.  The present study has replicated these findings and extended the field to different kinds of mindfulness manipulations and also to a long-term study.  Long-term research of this type is more costly to perform and sometimes harder to interpret, but is essential if psychologists are to establish a prescription for healthy though patterns.  

An ideal trajectory for further research is into religiosity, which has been only scarcely investigated in the past.  While Buddhism is not a religion in the strict sense, and the practice of mindfulness need not have any connection to spirituality, research indicates that different effects may be found in some areas if the practice is being viewed in a religious light.  

An even more rewarding direction for future research may be into the effects of intensive mindfulness trainings, which often come in the form of retreats for a week or more.  One study which examined this phenomenon found innumerable effects from a seven day Vipassana retreat at which participants were awoken at 4 in the morning, and meditated, walking, standing or sitting throughout the day in complete silence.  Such an experience could not fail to leave lasting memories and perhaps to make lasting and positive changes to thought patterns and behavior \cite{Emavardhana1997}.  Further research in this direction is desirable and possible.  

\section{Concluding Remarks}

The results of this study, building on the growing literature connecting mindfulness to self-esteem show a potentially powerful effect of mindfulness on reducing defensiveness in persons with defensive, high self-esteem.  These effects however were only present immediately after the training was complete if the training was a brief, one-time session, whereas if the training was a full course on mindfulness, the effects lasted for a full six months and beyond.  These results should be considered by clinicians when deciding how to lastingly improve their patients' lives, and by individuals seeking to improve their own self-esteem and other traits via mindfulness.  Future directions for research include other types of mindfulness practices and the possible differences between secular and religious mindfulness.  
\bibliography{422.bib}                                                                                                                                                

     
\end{document}




