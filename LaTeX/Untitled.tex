\documentclass[aps,pre,nofootinbib]{revtex4}

\usepackage{amsmath,amssymb,amsfonts,amsthm}
\usepackage{graphicx}
\usepackage{bbm}
\usepackage{pdfsync}
\usepackage{color}

\sloppy
\definecolor{lightgray}{gray}{0.5}
\setlength{\parindent}{0pt}

\begin{document}

\title{Matlab Problem Set II}

\author{Gray Davidson}
\affiliation{Department of Physics, Reed College, Portland, Oregon,  97202, USA}
\affiliation{For Physics 331, week 9}
\date{\today}

\maketitle

\begin{enumerate}
\item
   \begin{verbatim}
help sym
i
i = sym ('x')
i
A = sin(x)
int(A,x)
int(A,x,0,pi)
ezplot(int(A,x),[0,4*pi])
z = sym('z')
ezplot(erf(z))
\end{verbatim}

\item
\begin{verbatim}

uiimport AppendixA.xls

textdata

textdata =

   'Solar mass'             [1.9891e+30]    'kg'
   'Solar irradiance'       [      1365]    'W m^{-2}'
   'Solar luminosity'       [3.8390e+26]    'W'
   'Solar radius'           [ 695508000]    'm'

\end{verbatim}

etc.

The full list is present.  I am cutting it short in my \LaTeX report to streamline that report.

\item
  \begin{verbatim}
help sym
i
i = sym ('x')
i
A = sin(x)
int(A,x)
int(A,x,0,pi)
ezplot(int(A,x),[0,4*pi])
z = sym('z')
ezplot(erf(z))
\end{verbatim}

\includegraphics{sine}
\includegraphics{erfz}
\item

\begin{enumerate}

\item Only a ``d" is required to cause Matlab to use diamonds as the points on a plot.

\item THe result is instantaneous transportation to the documentation for Line Specification.  That is, the documentation on how to change a plotted line's properties such as color and width.

The string for adjusting marker size is `markerSize,' and the size is determined in `points.' Points are the default data points in Matlab.

\item

   \begin{verbatim}
[x, y] = meshgrid (-2.1:.15:2.1, -6:.15:6);
u = 80 * y.^2 .* exp(-x.^2 - 0.3*y.^2);
mesh (u)
\end{verbatim}

\includegraphics{doublehorn}

\item
n = 0:2000;
polar (pi*137.51*n/180,sqrt(n),'o')

\includegraphics{sunflower}

\end{enumerate}

\item

  \begin{verbatim}
g = 1:4:121;
k = 3:4:123;
x = sym ('x');
h = x.^g./g
i = x.^k./k
j = sum(h)-sum(i)

\end{verbatim}
       \color{lightgray} \begin{verbatim}
h =

[ x, x^5/5, x^9/9, x^13/13, x^17/17, x^21/21, x^25/25, x^29/29, x^33/33, x^37/37, x^41/41, x^45/45, x^49/49, x^53/53, x^57/57, x^61/61, x^65/65, x^69/69, x^73/73, x^77/77, x^81/81, x^85/85, x^89/89, x^93/93, x^97/97, x^101/101, x^105/105, x^109/109, x^113/113, x^117/117, x^121/121]


i =

[ x^3/3, x^7/7, x^11/11, x^15/15, x^19/19, x^23/23, x^27/27, x^31/31, x^35/35, x^39/39, x^43/43, x^47/47, x^51/51, x^55/55, x^59/59, x^63/63, x^67/67, x^71/71, x^75/75, x^79/79, x^83/83, x^87/87, x^91/91, x^95/95, x^99/99, x^103/103, x^107/107, x^111/111, x^115/115, x^119/119, x^123/123]


j =

x^121/121 - x^123/123 - x^119/119 + x^117/117 - x^115/115 + x^113/113 - x^111/111 + x^109/109 - x^107/107 + x^105/105 - x^103/103 + x^101/101 - x^99/99 + x^97/97 - x^95/95 + x^93/93 - x^91/91 + x^89/89 - x^87/87 + x^85/85 - x^83/83 + x^81/81 - x^79/79 + x^77/77 - x^75/75 + x^73/73 - x^71/71 + x^69/69 - x^67/67 + x^65/65 - x^63/63 + x^61/61 - x^59/59 + x^57/57 - x^55/55 + x^53/53 - x^51/51 + x^49/49 - x^47/47 + x^45/45 - x^43/43 + x^41/41 - x^39/39 + x^37/37 - x^35/35 + x^33/33 - x^31/31 + x^29/29 - x^27/27 + x^25/25 - x^23/23 + x^21/21 - x^19/19 + x^17/17 - x^15/15 + x^13/13 - x^11/11 + x^9/9 - x^7/7 + x^5/5 - x^3/3 + x

Which is arctan of x.

\end{verbatim} \color{black}
\begin{verbatim}
g = 1:4:1121;
k = 3:4:1123;
x = 1/8;
h = x.^g./g;
i = x.^k./k;
j = sum(h)-sum(i)
x = 1/57;
h = x.^g./g;
i = x.^k./k;
w = sum(h)-sum(i)
x = 1/239;
h = x.^g./g;
i = x.^k./k;
l = sum(h)-sum(i)
(6*j+2*w+l)*4
\end{verbatim}
       \color{lightgray} \begin{verbatim}

j =

   0.1244


w =

   0.0175


l =

   0.0042


ans =

   3.1416

\end{verbatim} \color{black}

\item

\begin{verbatim}
y = 0:.00001:1;
b = (1.+y).^(1./y);
e = 2.71828182845904;
r = b-e;
s = r (1:1000:99999);
s'

ans =

      NaN
  -0.0135
  -0.0267
  -0.0397
  -0.0524
  -0.0650
  -0.0773
  -0.0894
  -0.1013
  -0.1130
  -0.1245
  -0.1359
  -0.1470
  -0.1580
  -0.1687
  -0.1793
  -0.1898
  -0.2001
  -0.2102
  -0.2201
  -0.2300
  -0.2396
  -0.2491
  -0.2585
  -0.2678
  -0.2769
  -0.2859
  -0.2947
  -0.3034
  -0.3120
  -0.3205
  -0.3289
  -0.3371
  -0.3452
  -0.3532
  -0.3612
  -0.3690
  -0.3767
  -0.3843
  -0.3918
  -0.3992
  -0.4065
  -0.4137
  -0.4208
  -0.4279
  -0.4348
  -0.4417
  -0.4484
  -0.4551
  -0.4618
  -0.4683
  -0.4747
  -0.4811
  -0.4874
  -0.4936
  -0.4998
  -0.5059
  -0.5119
  -0.5178
  -0.5237
  -0.5295
  -0.5353
  -0.5409
  -0.5466
  -0.5521
  -0.5576
  -0.5630
  -0.5684
  -0.5737
  -0.5790
  -0.5842
  -0.5893
  -0.5944
  -0.5995
  -0.6045
  -0.6094
  -0.6143
  -0.6191
  -0.6239
  -0.6287
  -0.6334
  -0.6380
  -0.6426
  -0.6472
  -0.6517
  -0.6561
  -0.6606
  -0.6649
  -0.6693
  -0.6736
  -0.6778
  -0.6820
  -0.6862
  -0.6904
  -0.6945
  -0.6985
  -0.7025
  -0.7065
  -0.7105
  -0.7144
\end{verbatim}

  We can see that as the indices fall to 0 (upward on the list), the output approaches zero as well, which is the desired result, since we have subtracted off the value of e.

  \item the `type' command is used to initiate a scrips and it takes an input of the name of that script.

  arctan(x) which we had to calculate from scratch in 8.2 is an example of a built in function.

  The `why' function, which is admittedly just for fun, is an example of a function m-file.

  \item  I cant make the function m-files work for me.

  I input this:
\begin{verbatim}
  function gravity = gravity1(r)
m1 = 10
m2 = 1
G = 6.67300*10^11
gravity = G(m1*m2)/r^2
end
\end{verbatim}
  And then I run the names, and get this:

\begin{verbatim}

  >> r = 10

r =

   10

gravity1
??? Undefined function or variable 'gravity1'.

gravity2

m1 =

   10


m2 =

    1


G =

  6.6730e+11

??? Index exceeds matrix dimensions.

Error in ==> gravity2 at 5
gravity = G(m1*m2)/r^2
\end{verbatim}
\item

Again I'm stumped on the exponential question.

I tried it two ways
\begin{verbatim}

x = 1:15
y =
a = sym ('a')
b = a.^x./

a = sym ('a');
for n = 1:100;
   x = 1;
   x = a * x/n
   end
\end{verbatim}
   in the first case, I couldn't figure out how to incorporate the factorial, and in the second, I couldn't figure out how to sum across elements.

   \item

   Using Newton's code, (shown below)
\begin{verbatim}

steps = 0;
x = input( 'Initial guess: ' );
re = 1^(-8);
myrel = 1;

while myrel > re & (steps < 20)
   xold = x;
   x = x-f(x)/df(x);
   steps = steps + 1;
   disp( [x f(x)] )
   myrel = abs((x-xold)/x);
end;

if myrel <= re
   disp( 'Zero found at' )
   disp( x )
else
   disp( 'Zero NOT found' )
end;

And inputting the functions

function [ y ] = g( x )
y = x^3 - 2*x - 5


end

and

function [ y ] = dg( x )
y = 3*x^2 - 2


end
\end{verbatim}
But I still get back, as the script's estimate for the zero, exactly what I input as the initial guess, even when some simple algebra on my part is enough to see that the guess is not right.

For instance:

Initial guess: 2
Zero found at
    2

\end{enumerate}
\end{document}
