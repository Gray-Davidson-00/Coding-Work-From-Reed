\documentclass[aps,pre,twocolumn,nofootinbib]{revtex4}

\usepackage{amsmath,amssymb,amsfonts,amsthm}
\usepackage{graphicx}
\usepackage{bbm}
\usepackage{pdfsync}
\usepackage{color}
\usepackage{hyperref}


\begin{document}

\begin{figure}[h]
\centering
\includegraphics*[width=  2.0 \columnwidth]{seals} 
\end{figure}

\bigskip
\bigskip

\bigskip
\bigskip
\bigskip

\title{Junior Qualifying Exam 2011}

\author{Gray Davidson}
\affiliation{Department of $\Psi$, Reed College, Portland, Oregon,  97202, USA}
\date{\today}


\maketitle



\section{Background:}

	Connor Diemand-Yauman, Daniel M. Oppenheimer, and Erikka B. Vaughan, researchers at Princeton University begin their article ``Fortune favors the \textbf{bold} (\textit{and the Italic}): Effects of disfluency on educational outcomes" with a new  take on an old story, that is, the failure of intuition in the context of cognitive psychology.  On this occasion, the intuition in question is that a fluent and easy learning session, with as few distractions as possible adding to the cognitive load, will lead to better recall at a later date.  This intuition, although it is shared by both teachers and students, turns out to be theoretically and experimentally wrong since ease of encoding is not correlated to subsequent performance.  

	The aspect which is indeed correlated is depth of processing, and in particular, `deep' processing as suggested by (Craik \& Lockhart 1972) is correlated to better encoding and recall, while `shallow' processing although it acts faster leads to poorer recall.    This idea, combined with the concept of mental heuristics, contemporaneously suggested by (Kahneman \& Tversky 1973), forms one of the theoretical pillars upon which the target article builds its approach.  Heuristics are time-saving `shortcuts' which the brain may use in familiar situations because deeper processing is more costly in terms of time and mental effort.  A person using heuristics is therefore more likely to make errors.  The second pillar is the idea that this deeper processing can be plausibly stimulated via ``desirable difficulty," or in this case, disfluency.  
	
	They make the point that difficulty and disfluency are not the same thing.  Difficulty, so called, is objective difficulty, while disfluency is subjective difficulty, and importantly, disfluency can exist without difficulty.  There is in fact a third level to this puzzle as well, which will become important later, which is meta-cognitive bias regarding difficulty.  That is, research subjects will easily be able to tell researchers what task they think will be harder, but in several cases this prediction does not line up with reality, as in (Rhodes \& Castel's 2008) study of font size and in (Alter et. al. 2007) study of logical syllogisms in hard-to-read fonts.  
	
	Henceforth in this document, shallow processing will be referred to generically as `system 1' processing and deep processing will be referred to as `system 2' to borrow terminology from (Stanovich \& West 2000).  It should also be noted that these systems are colloquially referred to as intuition and reasoning respectively.  These terms are more mature than the original levels suggested in 1972, but certainly encompass those original concepts.
	
	The researchers responsible for the target article are motivated by a desire to improve the educational system in the United States, and due to this real-world goal, a question of great importance to them is generalizability.  The problem becomes obvious when in their first study, they employ alien taxonomies so as to control prior exposure to the stimuli, but this is yet another layer of artificiality laid over the laboratory setting.  Strangely, the opposite extreme is attained in their second, `real-world,' study.  In this case, they interfere in almost no way with an existing educational context, simply exchanging one small aspect of the environment for another similar one.  With one small caveat regarding a manipulation check, this study should serve as a model of noninterference between researchers and subjects.  
	
		It should be mentioned that shallow processing, the level we are here trying to avoid, is not categorically `bad,' just as deep processing is not categorically `good.' Indeed, if one were consistently desirable over the other, then one would certainly have been systematically eliminated by the slow march of the evolutionary process.  

	Shallow processing has a place, just as deep processing has, most commonly as a method of finding a rapid answer to some situation based on prior experience, without taking the time to evaluate all outcomes.  Time is a key issue here, along with effort, and so motivation is a factor.  Deep processing takes more time and effort but often yields more complete, more abstract and better understood results.  When in an academic setting, this is obviously a desirable outcome, but at times the advantages come with too high a price tag in terms of time and mental effort--the decision is sometimes made by the student with unfortunate consequences.  
	
	A final concern which the target article raises regarding this type of research (although the Diemand-Yauman et. al. study manages to avoid this pitfall) is that of ceiling and floor effects.  The latter is what (Rhodes \& Castel 2008) encountered when they tested font sizes which were all large enough that although participants expected different levels of difficulty, there was no difficulty either subjectively or objectively.   A ceiling effect threatened the studies in the target article, but was fortunately avoided.  If the fonts had been to small or too hard to read, the study would have been more akin to translating a foreign language, where not only would the reading seem hard, but every word might require individual attention and translation.  At this point, subjective fluency would be irrelevant since the act of reading would be objectively difficult enough to interfere with results.  The target article avoided this by using relatively common fonts (as opposed to something like gothic calligraphy), and by reducing type size only when legibility could be preserved through the process.  Altogether, they did a good job avoiding these issues, but further studies on this topic must be aware of them.  

\section{Review:}

	Simply stated, the hypothesis of the target article was that ``disfluency can function as a cue that one may not have mastery over material," and therefore, ``to the extent that a person is less confident in how well they have learned the material, they are more likely to engage in more effortful and elaborative processing styles."  
	
	They tested this hypothesis first in a laboratory and then in the real-life situation of high-school classrooms.  The first experiment (n = 27) involved memorization of taxonomical data.  Subjects were given 90 seconds to memorize 21 separate data, distracted for 15 minutes, and then tested on a random 7 of the items.  Results showed 72.8\% accuracy when the data were presented in a standard ariel front, 100\% black, and normal size.  Subjects showed 86.5\% accuracy when the data were in an unfamiliar font, at 60\% gray, and of smaller size.  The counter-intuitive 14\% increase in accuracy would seem to support the researchers' hypothesis.  
	
	In the second study, in the classroom, (n = 222) all variables were kept constant, except for the typeface of the various reading materials.  Neither the students, nor the teachers were made aware of the hypothesis.  The participants were enrolled in 12 different high-school classes: two sections each of AP English, Honors English, Honors Physics, Regular Physics, Honors US History, and Honors Chemistry.  One class of each pair was maintained on the traditional fonts used in that class, while the other was switched to a disfluent font for the duration of one testing period (a variable interval from 1.5 to 4 weeks).  

	The resulting design could have been a 2 (font difficulty) x6 (high-school subject) design, but ended up being a simple comparison of the two font conditions since the researchers chose to simply average the Z-scores for the six different classes in each font condition.  Arguably the 'difficult font' parameter had several sub-elements, but an analysis revealed no differences between comic sans, haettenschweiler and monotype corsiva, so these were combined, and a binary was used for the final analysis.  

	At this point in their analysis, the researchers seem to have passed over additional points which demand attention.  The fact that a t-test performed on the aggregate data was statistically significant is undeniably important and is the fuel for the remainder of their discussion (and certainly p < .001 is nothing to laugh at), but to examine Table 1 (the class-by-class breakdown table), we see that not only were some classes' results wildly stronger than others (to the tune of two orders of magnitude), but that in fact, the result for the chemistry class took the opposite direction from all the others, and from the aggregate.  

	What this means is that the statement which opens the discussion section: ``This study demonstrated that student retention of material across a wide range of subjects (science and humanities classes) and difficulty levels (regular, Honors and Advanced Placement) can be significantly improved in naturalistic settings by presenting reading material in a format that is slightly harder to read," is certainly misleading.  Without blaming the researchers' intention or inattention in particular, it should be noted that the chemistry class at the very least could not have supported this result, and therefore although the aggregate is certainly interesting for the myriad reasons noticed in the target article, there might easily be something else also occurring here, such as a different result based on the type of learning or the type of classroom materials presented in different subjects.  	

	One other area in which the target article fell short of ideal was in terms of manipulation checks.  The second study functions quite well as a test of the generalizability of the first, and in that second study, Diemand-Yauman et. al. did indeed administer a questionnaire at the close of their testing period asking students questions which targeted motivation, but in other areas, they chose to use a rational argument rather than a simple test to ensure their manipulation was airtight.  Given the simplicity of expanding their questionnaire, this simple edit would set the reader's heart at ease.

	Remarkably, Diemand-Yauman et. al. nearly managed, in a high-school classroom, to achieve the pharmacist's grail of double blindness as neither the teachers nor the students in their study were aware of the hypotheses involved.  The participants in this case were of course capable of distinguishing between the several conditions, but still were not told what was expected from any of them.  

\section{Proposal:}

	One possible direction to head when expanding on the target article's research is toward a top-down induction of disfluency, for instance by simply telling students that the coming material is comparatively difficult.  Unfortunately, (Rhodes \& Castel 2008) inadvertently made a prediction about the results of this study.  Their null result indicates that the meta-cognitive bias of the subject is not sufficient to induce subjective disfluency alone.  Participants will expect difficult material but will not actually perform better because of that expectation.  
	
	Another possible direction to head would be to attack the learning difference between chemistry and other subjects.  Especially in light of the differences between chemistry and other subjects also reported in (Lehman, Lempert \& Nisbett 1988), there seem to be large cognitive differences between chemistry and other studies, but the subtlety of teasing chemistry apart from regular physics is likely beyond the sensitivity of the desirable difficulty design.  

	The motivation for the research in the target article was a desire to improve the field of education with a very cost-effective switch from traditional, easy-to-read fonts to less common, slightly less intuitive fonts so as to jar students into system 2 processing.  

	Applying the same reasoning, there are other areas of the educational system which might be benefited by research in this field.  One of these is the traditional separation of reading levels between different high school classrooms.  Given the results of the target article, it is possible that by assigning easier reading material to students who are already rated as disadvantaged, educators are further widening the learning gap between these students and those in more advanced classes.  

	A study is therefore proposed which would test the intersection of differential text levels across different high school english classes containing students at different levels.  The first challenge, although not a tremendously difficult one is to find, somewhere in the U.S. a high-school teacher in charge of four sections of english classes, two at an advanced level and two at a 'remedial' level.  While this exact setup may be uncommon, it almost certainly does occur at some U.S. school, so this design should not present any real obstacle.  

	The design then is a 2 (advanced or remedial english class) x2 (reading level of assigned readings).  Each cell of this 2x2 will be one high-school classroom (15-25 students) for a total of 60-100 participants.  

	The reading for the course will be from the New Testament, read in the context of literary analysis and not theological studies, with the justification that translations thereof effectively form a continuum since there are so many of them.  (\href{http://www.christianbook.com/Christian/Books/cms_content?page=652502&sp=1003}{See Christianbook.com}).  In particular, the King James Version and the Common English Bible will be used.  The first of these is rated at a 12th grade reading level, and the second at a 7th grade level, although a simple preliminary test could reveal whether these are the appropriate levels, and if necessary another translation substituted.  

	What is predicted from this is that those students presented with a more complicated text will do better on later exams than the students with the simpler text.  Obviously it is predicted that the honors english classes will do better than the remedial along each column of the design, but there is to be no prediction about the relationship between the diagonal cells.  Here is where the really interesting results will be found.  One diagonal is the classic structure: the better class reads harder material, which replicates the national situation and will act as a control comparison, but the other diagonal has the opposite reading assignments, and it remains to be seen whether the remedial class can actually surpass the honors class or not in this case, given the advantage of more complicated reading to induce system 2 processing.  

	As opposed to the real-world study in the target article, this study will need to interfere in several ways with the existing curriculum in the chosen school.  The reading will be chosen for the four classes as mentioned above, and additionally, the same test must be given to all four classes and must be administered after the same period of study.  In this case, as in the target article, it would be prudent to avoid telling either students or teachers anything about the study, but unlike that article, the teachers in this case have four cells for comparison, and may be more likely to reconstruct the study hypothesis.  Although the logic of the target article does still apply however, that they would be unlikely to predict the directionality of the result correctly.  

	With students in early high-school as subjects, the problems of ceiling and floor effects will be avoided by the choice of texts, using texts rated for the extremes of secondary school reading (7th and 12th grades), it is hoped that they will neither be unreadably complicated, nor trivially easy.  As noted above, a short preliminary study should be conducted with a separate but similarly situated group of students to find out what translation is appropriate to avoid these pitfalls.  The complexity of the reading level is determined by complicated algorithms, involving the average number of characters and syllables per word, the average length of sentences, and the types of grammatical constructions frequently used in the work.  For instance, ``nominalizations, misplaced phrases, relative pronoun and compliment deletions�and the use of passive voice in subordinate classes," to name a few (Charrow \& Charrow 1979, cited in Lind 1982).  

	The question of motivation is a serious one here, and this is the most likely realm for error in this study.  When presenting a remedial english class with reading above its level, there is potential for students to lose motivation, and perform poorly not due to system 1 processing but simply because they have not completed the reading or not taken the time to comprehend it.  This problem was not present in the target article although Diemand-Yauman et. al. did administer a questionnaire at the close of their study to assess any differences in student motivation.  Students in a remedial english class also cannot help but feel themselves judged negatively before they even start reading, and their membership in a marginalized category can negatively affect their normal performance (Shih et. al. 1999).  This is not a factor which can be avoided, but should be kept in mind.

	A similar questionnaire is therefore proposed in the present study, not only at the end of the testing period, but also at the beginning and at one intermediate time to attempt to assess changes in motivation.  In addition to the four questions used by Diemand-Yauman et. al., one additional question will also be appended to the final edition of the questionnaire, reading ``what do you think was the purpose of this study" which will be followed by a blank space for an answer, rather than a 5-point scale.  This is the manipulation check lacking in the target article, and should be administered not only to students, but also to the teacher as one of the first parts of a debriefing interview.  

	This study differs from the target article in the method of presentation of the desirable difficulty (and in particular, brings that presentation into the more realistic framework of reading level).  All the studies on this topic mentioned in the target article (Rhodes \& Castel 2008, Alter 2007 etc.) as well as the target article itself used purely visual manipulations to increase difficulty whereas manipulations of the complexity of the stimuli are arguably more salient in a scholastic environment.

	If the results of the study show that students in remedial english classes lose motivation detrimentally when presented with harder reading, then an alternative method may need to be found to challenge and engage them academically.  If students in the remedial condition perform poorly with hard readings, then the intuitive system would seem to be correct and should not be changed, but in the case that disadvantaged students perform better with hard readings, and especially if they out-perform the advanced class with easy readings, then support is demonstrated for curricular changes on a wide scale just as it is in the target article.  
	
	\begin{thebibliography}{99}
\bibitem  SShih, M., Pittinsky, T. L., \& Ambady, N. (January 01, 1999). Stereotype Susceptibility: Identity Salience and Shifts in Quantitative Performance. Psychological Science, 10, 1, 80-83.
\bibitem  LLehman, D. R., Lempert, \& R. O., Nisbett, R. E. (January 03, 2002). The Effects of Graduate Training on Reasoning: Formal Discipline and Thinking About Everyday-Life Events.  American Psychologist, 43, 6, 431-442.
\bibitem  LLind, E. A. (1982). The Psychology of the Courtroom.
\bibitem  CCraik, F. I., \& Lockhart, R. S. (1972). Levels of processing: A framework for memory research. Journal of Verbal Learning \& Verbal Behavior, 11(6), 671�684.
\bibitem  RRhodes, M. G., \& Castel, A. D. (2008). Memory predictions are influenced by perceptual information: Evidence for metacognitive illusions. Journal of Experimental Psychology: General, 137, 615�625.
\bibitem  CCharrow, R. P., \& Charrow, V. R. Making Legal Language Understandable: A Psycholinguistic study of jury instructions.  Columbia Law Review, 1979, 79, 1306-1374.
\bibitem  AAlter, A. L., Oppenheimer, D. M., Epley, N., \& Eyre, R. (2007). Overcoming intuition: Metacognitive difficulty activates analytic reasoning. Journal of Experimental Psychology, 136(4), 569�576.
\bibitem  KKahneman, D., Slovic, P., \& Tversky, A. (1982). Judgment under uncertainty: Heuristics and biases. Cambridge: Cambridge University Press.
	\end{thebibliography}

\end{document}
