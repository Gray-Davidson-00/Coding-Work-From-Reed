\documentclass[aps,pre,twocolumn,nofootinbib]{revtex4}

\usepackage{amsmath,amssymb,amsfonts,amsthm}
\usepackage{graphicx}
\usepackage{bbm}
\usepackage{pdfsync}
\usepackage{color}

\begin{document}

\title{A Measurement of the Leidenfrost point in ethanol And Water}

\author{Gray Davidson}
\affiliation{Department of Physics, Reed College, Portland, Oregon,  97202, USA}
\affiliation{For Physics 331, week 8}
\date{\today}

\begin{abstract}  
The Leidenfrost Effect is explored, and the Leidenfrost point determined experimentally for water and ethanol on a machined aluminum surface.  For the latter, it lies at $~130 +/- 5\,^{\circ}{\rm C}$ and for the former, at $~140 +/- 5\,^{\circ}{\rm C}$.  Numerous sources of error and sparseness of data points prevented greater certainty than this.  The inflection point in the mean lifetime of liquid droplets on hot surfaces was also confirmed.  

\end{abstract}
\maketitle

\section{Introduction}
Ordinary life carelessly spreads the most complex mysteries before us every day.  In the present experiment, greater understanding is sought for an effect any chef knows well.  When a liquid, (such as water) is dripped on a surface which is maintained above its boiling temperature (such as a frying pan), the liquid often takes far longer to boil away than intuition tells us it should--if a pot full of drops can boil away in an hour, surely a single drop should take no more than a few seconds.  Intuition proves correct for temperatures close to the boiling point of the liquid, (thus at 110 $\,^{\circ}{\rm C}$, water droplets boil away in less than a second, but fails when temperatures extend into a new regime, beginning at what is called the Leidenfrost point, (near 140 $\,^{\circ}{\rm C}$ for water).  In this regime, water droplets, or droplets of other liquids (ethanol was also used in the present experiment) can survive, seemingly in contact with hot surfaces for periods as long as minutes.  The explanation, \cite{Self_Propelled_Leidenfrost_Droplets} is that the lowest layer of liquid molecules is vaporizing and then insulating the remainder of the liquid from the heated surface.  

The effect in question, which promotes longevity, is called the leidenfrost effect, named for Johann Gottlob Leidenfrost, (1715 - 1794).  It was first discovered in 1756 by the following quotated experiment: 

"An iron spoon of any size, well polished within and free from rust and dirt, is heated over glowing coals until it glows with light.  To this glowing spoon, removed from the coals, send through a glass tube of suitable length, of which the other end finishes in a very narrow capillary canal, one drop of very pure distilled water....At the instant when the drop touches the glowing iron, it is spherical. It does not adhere to the spoon, as water is accustomed to do, which touches colder iron....If then the spoon remains motionless, this water globule will lie quiet and without any visible motion, without any bubbling, very clear like a crystalline globe, always spherical, adhering nowhere to the spoon, but touching it in one point. However, although motion is not visible in the pure drop, nevertheless it delights in a very swift motion of turning, which is seen when a small coloured speck, for example some black carbon, adheres to the drop. For this is turned around the drop with a wonderful velocity.... Moreover, however this drop only evaporates very slowly." \cite{original}

The Leidenfrost effect is responsible for several common amusements from liquid nitrogen or from dry ice, which will exhibit the same properties at room temperature that alcohol or water might above their respective boiling points.  Intrepid performers have at times even trusted the effect to protect them when blowing liquid nitrogen out of their mouths.  Others have used it to protect their skin from the heat of molten lead, simply by wetting their hands before plunging them into the metal.  These effects do indeed work, and are impressive to behold, but more impressive by far is another point in which the Leidenfrost effect runs counter to the common intuition.  

When the surface in question is not a smooth surface, but it perforated in a regular sawtooth pattern, Fig. ~\ref{sawtooth} \cite{sawtooth1} The water will flow in a certain direction related to the shape of the machined perforations.  This effect can be used to keep the water droplet in a circular orbit, or to force it to flow uphill.  Of course the water's gain of gravitational potential does not actually represent a violation of conservation of energy because the water drop is taking in energy in the form of heat, from the heated plate.  \cite{Self_Propelled_Leidenfrost_Droplets, cindy}

\begin{figure}[h]
\centering
\includegraphics*[width=  .9 \columnwidth]{sawtooth} 
\caption{The water droplet's direction and speed can be controlled by properties of the heated surface.  The water can be made to flow uphill.}
\label{sawtooth}
\end{figure}

The Leidenfrost effect can at times be much more serious however, than these \textit{popular science} demonstrations, as in the case of nuclear reactors.  Many reactors are cooled by the flow of water past their control rods, and depend on the contact between the water and this metal surface to prevent meltdowns induced by extreme temperatures.  

Nucleate boiling is the process by which many reactors are cooled--vapor bubbles forming on impurities of the metallic surfaces carry away the energy from the reactor core.  When there is too much energy being produced, the small bubbles join together, forming a sheath of water vapor around the fuel elements which insulates them from the flowing coolant.  The temperature of the surface must then be much higher to maintain a standard heat flux into the cooling fluid, and if this temperature is higher than the melting point of, say, aluminum, emergencies may ensue, such as meltdowns.  

This second regime, (the one of interest) is superseded by a third, in which the temperature is simply too great for the Leidenfrost effect to have full effect.  In this regime, the time taken for a liquid droplet to boil away decreases slowly, meaning that there is a maximum lifetime for such a droplet.  The point at which its lifetime is a maximum is the point which the current experiment searched for, as well as the Leidenfrost point itself, although the latter proved difficult to identify since the method was less and less accurate in this temperature range.  

\section{Methods and Apparatus}

The apparatus for the current experiment was relatively simple, consisting of an aluminum block with three notable features.  The first was a machined cavity in its top, as smooth as possible, and with beveled sides so as to keep a water droplet on the block.  The second was a resistive heating element inside the block, about 1.5 inches distant from the cavity.  The third was a thermocouple to measure the surface temperature of the block.  The apparatus was attached, via the thermo-couple's leads and the leads from the heating element, to an interface which would take a user-defined target temperature and asymptotically approach that temperature by sending a modulated current through the heating element, although any apparatus capable of holding a plate at a constant temperature could be used as well.  

A standard chemical eye-dropper was used to drip water and ethanol onto this aluminum block.  No additional apparatus was used for the experiment itself, although a balance sensitive to .1 mg was used to determine the standard size of a water droplet.  This size was determined to be $0.023 +/- .004 g$.  

The respective boiling points of ethanol and water are 80 $\,^{\circ}{\rm C}$ and 100 $\,^{\circ}{\rm C}$ respectively, so no data was collected below 80 $\,^{\circ}{\rm C}$ where the mean lifetime was assumed to be determined by natural evaporation.  

Additional preliminary tests determined that despite Leidenfrost's original method: ``This drop which first fell upon the glowing iron is divided into a few little globes, which nevertheless after a little while are collected in one great globe again," \cite{original}a better way to get more consistent drop size was to drop water or ethanol from as close to the surface as possible so as to minimize splashing.  Leidenfrost's main concern was the effect itself, while the present experiment aimed to determine properties of the materials and liquids in question.  

Among the important observations was that droplets of too large a size were not stable on the heated plate.  These would oscillate wildly and deform randomly, often hissing as they encountered imperfections in the machined aluminum.  When this hissing sound was heard, it was determined qualitatively that the droplet was evaporating much faster than it would have given the leidenfrost effect.  The rate of evaporations would remain high until the droplets were of a standard size, small enough to avoid these effects.  Conversely surface tension prevented the formation of droplets of too small a size.  

Additional, unavoidable effects were observed at different temperatures as the plate was heating.  Firstly, boiling was not consistently achievable until the plate was kept at 110 $\,^{\circ}{\rm C}$ (in the case of water) and 90 $\,^{\circ}{\rm C}$ (in the case of ethanol), which leads to suspicion of the thermocouple.  In the range between 110 $\,^{\circ}{\rm C}$ and 130 $\,^{\circ}{\rm C}$, the boiling droplet would often spit off many sub-droplets, which would boil separately on other parts of the plate.  In the range from 130 $\,^{\circ}{\rm C}$ to 150 $\,^{\circ}{\rm C}$, measurements ranged significantly due to the equilibrium location of the droplet as it was evaporating.  In all cases, the droplet would move around the cavity, and settle in a single location, where it would finish its evaporation.  Depending on this location (there seemed to be several preferred ones) the droplet would sometimes spin, sometimes spit, and sometimes float perfectly still, noiselessly, which was the indication that the leidenfrost effect was dominating.  In this range of temperatures, again, it was noted that droplet s would often spit for a while, then, being of a standard size, would settle in one of these equilibrium locations and live out their evaporation while sitting stationary.  

the data collected were a simple set of ordered pairs, lifetime (of both alcohol and water) as the variable dependent on temperature of the plate.  In pre-experiment tests, normal tap water was used, but dissolved salts were found to collect on the plate, and these small chunks would often increase the amount of undue oscillation and spitting.  Distilled water was therefore used throughout the actual experiment, and this problem was thereby corrected.  

\section{Results}

The experimental results are displayed in appendix A, but the relevant summary, \textit{i.e.} a report of the mean lifetimes for each temperature are displayed in Tbl. ~\ref{data}.  A general increasing and then decreasing trend, as predicted, can be seen, although this is a much more stable pattern in the case of ethanol.  Many of the problems mentioned in the Methods section above were unique to the water droplets, and in general, ethanol was much better behaved.  Each mean displayed on the table is the result of at lest three individual measurements takes sequentially with the plate at the same temperature.  

\begin{table}[h]
	\caption{Experimental averages, based on three or more readings each.  See Appendix A for the actual experimental data.  It can be easily seen from these data that the average lifetime of both water and ethanol increased dramatically at the Leidenfrost point, then decreased slowly afterward. ML = Mean Lifetime.}
\begin{ruledtabular}
	\begin{tabular}{ccccc} 
	
	Temperature ($\,^{\circ}{\rm C}$) & ML Water (s) & ML Ethanol(s)\\  \hline
	85 & N/A & 6.90\\
	90 & N/A & 4.05\\
	95 & N/A & 1.08\\
	100 & N/A & 0.69\\
	105 & N/A & 0.735\\
	110 & N/A & 0.59\\
	115 & 17.40 & N/A\\
	125 & 4.64 & 5.86\\
	135 & 5.67 & 22.03\\
	145 & 70.04 & 31.40\\
	155 & 99.46 & 36.80\\
	160 & 61.34 & 34.13\\
	165 & 98.67 & 38.90\\
	175 & 94.36 & 39.38\\
	185 & 107.8 & 36.89\\
	195 & 100.55 & 38.94\\
	205 & 96.91 & 37.65\\
	215 & 119.48 & 33.74\\
	225 & 101.49 & 34.97\\
	235 & 102.59 & 33.48\\
	245 & 107.92 & 32.10\\
	255 & 106.22 & 30.69\\
	265 & 104.55 & 26.95\\
	275 & 92.12 & 26.55\\
	285 & 94.11 & 26.23\\
	\end{tabular}
	\end{ruledtabular}
	\label{data}
\end{table}

For a visual representation, the same data are displayed in Fig.~\ref{graph}, where the instability of the water trend is evident, alongside the ethanol trend.

\begin{figure}[h]
\centering
\includegraphics*[width=  .9 \columnwidth]{graph.jpg} 
\caption{This is a graph of the data in Tbl. ~\ref{data}.  The dramatic increase in mean lifetime can clearly be seen at around 140 degrees.}
\label{graph}
\end{figure}

\section{Discussion}
As is clear from Fig.~\ref{graph} and also from the visible jumps in Tbl.~{data}, the leidenfrost point of water is between 135 $\,^{\circ}{\rm C}$ and 145 $\,^{\circ}{\rm C}$, while that of ethanol lies between 125 $\,^{\circ}{\rm C}$ and 135 $\,^{\circ}{\rm C}$.  This is the best approximation that can be made given the sparse nature of the data points in question.  Also visible, (fro the graph especially) is the inflection point, at which the leidenfrost effect is no longer effective in keeping the droplet from evaporating.  The surface is so hot in these temperature ranges that the heat flux across the insulating vapor will still boil the droplet away.  

One interesting section of the graph, which calls the experimental methodology into question is the initial peak, near the presumed asymptote at the boiling point.  It is proposed that this tail is the remainder of the asymptotic approach.  No data could be taken closer to the boiling point because the droplets were observed not to be boiling at all within about 5 $\,^{\circ}{\rm C}$ of that point.  It is also possible that the refusal to boil represents an uncalibrated thermocouple, which would imply that all readings taken are on the order of 5 $\,^{\circ}{\rm C}$ higher than the actual temperature.  

The erratic nature of the data, especially in the case of water may be attributed to the various concerns raised in the methods section above.  The droplet size is one issue which needs correction, and which may have an easy fix.  The proposed method in future experiments is a bulb of water held stationary above the plate.  When liquid is to be added, a set number of ml should be released from the reservoir via a graduated pipet.  This will hopefully standardize water droplet size, although problems may arise involving surface tension, thus total water dropped may need to be quantized in terms of total drops.  In any case, using 2-3 drops, instead of 1 will decrease the relevance of the size effects, and will still not be enough to severely increase the oscillations and deformation effects discussed above.  A procedure may be found in \cite{cindy} both for determining droplet mass, and also for creating droplets of approximately equal mass.  

Liquid droplets should be dropped from a standardized height as well, since even the variation of 1-2 cm drop height in the present experiment may have had an effect on the data.  

The present experiment provides a good springboard for future replication with similar apparatus, since the approximate values of the Leidenfrost and inflection points have been found.  Future scientists will need to take finer measurements in the vicinity of  these points to determine them more exactly.  To this end, a more accurate measure of temperature is recommended.  

\section{Conclusion}

Although a numerical answer more accurate than the approximation above: $~130 +/- 5\,^{\circ}{\rm C}$ for Ethanol, and $~140 +/- 5\,^{\circ}{\rm C}$ for water seems to be out of reach in the present case, the current experiment may serve as an excellent starting place for future refinement.  Several aspects of the Leidenfrost effect have been confirmed, both the sudden start of the effect, and also its gradual decline at higher temperatures.  Further research might further explore the associated Ratchet effect, or might explore the different Leidenfrost characteristics of different metal plates.  

	\begin{thebibliography}{99}
\bibitem{sawtooth1} BBC News: Scientists make water run uphill.  By Roland Pease http://news.bbc.co.uk/2/hi/science/nature/4955398.stm.
\bibitem{Self_Propelled_Leidenfrost_Droplets} H. Linke, B. J. Alema, L. D. Melling, M. J. Taormina, M. J. Francis, C. C. Dow-Hygelund, V. Narayanan, R. P. Taylor, and A. Stout, Phys. Rev. Lett.  \textbf{96}, 154502 (2006) 
\bibitem{original} J.G. Leidenfrost, A Tract About Some Qualities of Common Water, Ch. 15: On the Fixation of Water in Diverse Fire.  1756, University of Duisburg Press, (Available at: http://volcaniclightning.tripod.com/leidenfr.htm)
\bibitem{cindy} C. Joe, in Hovercrafts and Liquid Rafts: Extensions of the Leidenfrost Ratchet Effect, (Reed College Publishing, 2008).
	\end{thebibliography}

\end{document}