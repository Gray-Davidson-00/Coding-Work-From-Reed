\documentclass[jou]{apa}

\usepackage{graphicx}
%\usepackage{natbib}
\usepackage{pdfsync}

\title{Domain and Gender Effects on Others' Perceptions of Self-Handicapping Behavior}
\author{Davidson, G., Rodriguez, S., Sitney, M., Wise, A.}
\affiliation{Department of Psychology\\ Reed College}




\abstract{Past research has shown that participant-sex and target-sex moderates perceptions of self-handicapping behavior. The present research examined whether the domain in which the self-handicapping occurs is also a moderating variable. Specifically, if men and women rated targets differently when that target was engaged in a domain that is stereotypically male or female oriented. Results partially supported our hypotheses. Men supported self-handicappers more when the target was involved in a science class. Women supported self-handicappers more when the target was in an English class.  Additionally, all participants wanted the female target as a friend more than the male target when they were associated with the domain of science. Finally, English majors perceived targets in a science domain as having lower future potential than did other majors. These results indicate that the self-handicapping literature might benefit from considering the perceived gender of the domain that they are exploring.}

\acknowledgements{Gray Davidson, Sal Rodriguez, Miranda Sitney, Adrienne Wise, Department of Psychology, Reed College, Portland, Oregon.


Correspondence concerning this article should be addressed to the Social Self class (Psychology 422) of Kathryn Oleson, Reed College Psychology Department , 3203 SE Woodstock Blvd. Portland, OR.  97202.  E-mail: koleson@reed.edu}

\rightheader{ J. Personality \& Soc. Psych. 101 (2011) 526 - 531}
\shorttitle{Short title for manuscript header}
\leftheader{J. Personality \& Soc. Psych. 101 (2011) 526 - 531}



\journal{Journal of Personality and Social Psychology}
\volume{2011, Vol. 101, No. 6, 526 - 531}
\ccoppy{Copyright 2011 by the American Psychological Association}
\copnum{0022-3513/07/\$12.00 \quad DOI: 10.1037/0022-3513.92.3.527}



\begin{document}
\maketitle




\section{Introduction}
\

Self-handicapping is ''an individual�s attempt to reduce a threat to self-esteem by actively seeking or creating inhibitory factors that interfere with performance and thus provide a persuasive causal explanation for failure� \cite{Arkin1985}.  It can be conceived as a means of coping with self-doubt by providing an excuse for potential failure by blurring the lines between performance and ability \cite{oleson2010}. 
Self-handicapping has been extensively studied in academic contexts \cite{oleson2010}. Students who self-handicap engage in a variety of detrimental behaviors including not studying before an exam and going to a party instead so that low performance can be attributed to the party and not the ability of the student. This behavior has been found particularly among those with a high contingency of self-worth in academics with incremental theories of intelligence who self-handicap in the face of self-doubt about ability \cite{niiya2010}.

	Perceptions of self-handicapping have received less attention. Prior research has found that female perceivers of self-handicapping are more critical towards the behavior, with self-handicappers being viewed as more unmotivated and the motives of the self-handicapper viewed more negatively \cite{hirt2003}.
	
	In the Hirt (2003) paper, one study examined students perceptions of self-handicapping following the reading of a vignette. Participants read a vignette about a student who either self-handicapping or not the night before an exam and were asked to evaluate the target. The student either received an A or a D on the exam. The target was named Chris, with the gender of ``Chris� being varied.  Participants were asked to evaluate the target on a number of variables using a 1-6 scale-including evaluations of likability, ability, and future potential. To their noted surprise, no effects were found for the gender of Chris. Other findings were in line with prior research, with self-handicappers who did poorly on the exam being the most disliked. For those that performed well on the exam, while perceptions of ability and future potential were either higher or indistinguishable from controls (who studied rather than self-handicapped), there were also negative general perceptions of those who engaged in self-handicapping behavior.
	
	An additional study examined perceptions of the motives of the self-handicappers reading identical vignettes as the prior study. Motives included ``He is unmotivated and just doesn�t care about his grade in the class� and ``He was getting too stressed out/anxious about the test and needed a study break.� As with the first study, female participants indicated more negative perceptions of the target, and were noted to be more skeptical of the motives of the student. 
	
	This research seeks to retest what Hirt et al. had failed to find in terms of effect of target gender in perception of self-handicapping behavior. Unlike the Hirt et al. study, gendered domains (science and English) will be utilized to clearly make salient gender-related expectations. Further, clearly gendered names will be utilized to make stronger the effect of gender. It is our hypothesis that subjects will rate more positively those who self-handicap in a domain in which they aren�t expected to do well (for example, males in English) than those who self-handicap in their respective gendered domain (males in science, females in English)




















\section{Methods} 
\subsection{Participants}
	Participants were 244 undergraduate students at Reed College (70 men, 164 women). 10 students either chose an alternate gender or preferred not to answer and thus were excluded from the study for a total sample of 234 students. 35\% of the sample were mathematics or natural science majors, and 17\% were literature or language majors. All participants received one psychology lottery ticket for participation with the chance to win \$150.00. 

\subsection{Materials}
As our vignettes were designed to make the target sex/domain stereotypes salient, we felt it necessary to choose names for our targets about which our participants would not make biased intelligence judgments. Thus, referencing \cite{bradley2002}, Kristen was chosen as the female target�s name and Michael was chosen for the male target. 
There were eight distinct vignettes concerning self-handicapping behavior (see appendix A). All vignettes were primarily concerned with a student who has an exam the following day. In light of  \cite{hirt2003} findings, vignettes varied on the following dimensions: sex of target (male, female), self-handicapping (yes, no), and our critical manipulation, domain of exam (English, science).  In order to throw participants off from our true manipulations, the vignettes also varied on location of self-handicapping behavior (movie, friend�s house). This manipulation was not expected to have any effect on participants� perceptions and thus was ignored in the analysis. Thus, this study was a 2(target sex) x 2(self-handicapping) x 2(domain) x 2(participant gender) experimental design. 
Following each vignette was a set of questions (see appendix B). Two items assessing the participant�s attributions about performance and ability and seven items assessing interpersonal reactions to the target accompanied all vignettes. These items were rated on a 7-point Likert scale ranging from 0(poor) to 6(excellent). Finally, participants were presented with a seven-option question concerning their perceived motivations for the target�s behavior. In the self-handicapping trials, two of these items (�he/she wants to have an excuse for failure� and �he/she wants to show off possible success�) were directly taken from the \cite{hirt2003} to evaluate perceptions of self-handicapping. In non self-handicapping conditions, participants were given seven options with the opposite valence of the self-handicapping trials (e.g. �he is unmotivated� became �he is motivated�). These answers were only provided for consistency between conditions and were discarded from analysis. 

\subsection{Procedure} 
	Following an informed consent form, participants were randomly assigned to one of eight conditions. Each participant read two vignettes and answered the complete set of questions about each of them. For each participant, the pair of vignettes consisted of a primary vignette, and its opposite in terms of gender of target, self-handicapping condition, and domain of performance.  Following the vignettes, participants were asked to note their gender, division of their major, semesters in attendance at Reed College, and where they had heard about the study. Participants were then entered into the lottery and thanked for their time. 



\section{Results}
	The first and second vignette that participants read and rated were very similar to each other, which might have caused confusion or suspicion in participants. Therefore, results were analyzed to determine if participants� ratings on the second vignette appeared to be affected by having been exposed to the first vignette. Analyses revealed that position of vignette was important, such that there were significant position interaction effects with other variables. In other words, the overall pattern of results was markedly different for ratings made on the second vignette than the first. It was therefore concluded that ratings on the second vignette were unreliable and all further analyses only include ratings for the first vignette that participants saw.

A 2 (Target Gender: male vs. female) x 2 (Domain: math vs. science) x 2 (Self-handicapping: self-handicapping vs. non self-handicapping) x 2 (Gender of participant: male vs. female) ANOVA was conducted to assess each of the nine dimensions on which participants rated the target (see questions 1-9, the Performance and Ability Questions and the Interpersonal Questions, in Appendix B). A main effect of self-handicapping was found for several dimensions. When the target did not self-handicap, participants were more sympathetic towards them ($F(1, 217)= 19.30, p< .01$), wanted to have them more as a class partner ($F(1, 216)= 77.65, p< .01$) or a roommate ($F(1, 213)= 7.65, p< .01$), and rated them as having more future potential in their domain ($F(1, 218)= 6.02, p< .02$) than when the target self-handicapped. Finally, when the target did self-handicap, participants predicted that they would have done better on their exam if they had stayed home and studied instead ($F(1, 218)= 6.81, p< .01$). Thus, non self-handicappers tended to be viewed more favorably than self-handicappers were, at least in terms of the social variables less associated with ability. 

The 4-way ANOVA revealed additional interactions that are both comparable and contrasting to previous findings. A significant two way interaction between self-handicapping and gender of participant for participants� desire to have the target as a friend ($F(1, 215)= 7.50, p< .01$), shown in Table 1, and as a roommate ($F(1, 213)= 5.02, p< .03$) supports previous research. Simple mean comparisons for self-handicapping revealed that women participants wanted to be friends (M= 4.23, SD= .14) and roommates (M= 4.22, SD= .15) with targets who did not self-handicap more than targets who did self-handicap (friend: $M= 3.58, SD= .13$, roommate: M= 3.29, SD= .14). Thus, women�s views were more positive towards non self-handicapping behavior, mirroring the findings of Hirt, McCrea and Boris (2003). 

\begin{figure}[h]
\centering
\includegraphics*[width=  .9 \columnwidth]{table1.png} 
\caption{Self-handicapping x Gender interaction for desire to be friends with the target.  Note: For SelfHandicapping 0= self-handicapping target, 1= non self-handicapping target, Gender (of participants) 1= male, 2= female.}
\label{table1}
\end{figure}


A significant three-way interaction between target gender, self-handicapping, and gender of participant was found for participants� ratings of the target on liking ($F(1, 215= 8.91, p< .01$), sympathy ($F(1, 217)= 6.83, p< .01$), relation ($F(1, 215)= 4.78, p< .03$), and desire to have them as a friend ($F(1, 215)= 6.15, p< .02$), class partner ($F(1, 216)= 5.71, p< .02$), or roommate ($F(1, 213)= 6.00, p< .02$). Table 2 presents the results for reported sympathy towards of target. Replicating the findings of Hirt, McCrea and Boris (2003), several results indicate that women have particularly sensitive views of self-handicapping behavior. Results also indicate that women tend to favor other women who do not self-handicap the most, an effect of target gender that was absent in the work of Hirt, McCrea and Boris (2003). Simple mean comparisons for gender of target revealed that when the target did not self-handicap, women liked the target more if the target was female (M= 4.55, SD= .19) as opposed to male (M= 4.03, SD= .16). When the target did self-handicap, if they were female women wanted them less as a friend (M= 3.25, SD= .20) and a class partner (M= 1.76, SD= .22), than if they were male (friend: M= 3.90, SD= .18, class partner: M= 2.43, SD= .20). Similarly, simple mean comparisons for self-handicapping revealed that when women participants saw a female target who did not self-handicap, they liked (M= 4.55, SD= .19) and sympathized (M= 5.09, SD= .27) with her more and wanted her more as a friend (M= 4.27, SD= .22), class partner (M= 4.16, SD= .24), and roommate (M= 4.23, SD= .22) than if the target had self-handicapped (liking: M= 3.82, SD= .18, sympathy: M= 3.61, SD= .24, friend: M= 3.25, SD= .20, class partner: M= 1.76, SD= .22, roommate: M= 3.04, SD= .21). However, when women participants saw a male target that did not self-handicap, they were more sympathetic (M= 5.00, SD= .23) towards him and wanted him more as a class partner (M= 3.66, SD= .20) and roommate (M= 4.22, SD= .19) than if he did self-handicap (sympathy: M= 4.18, SD= .23, class partner: M= 2.43, SD= .20, roommate M= 3.55, SD= .19). The pattern for male participants was not as consistent, likely due to a small sample size of male participants. Thus, women were shown to view self-handicapping as an undesirable academic behavior overall, and have particularly negative views towards self-handicapping women.

\begin{figure}[h]
\centering
\includegraphics*[width=  .9 \columnwidth]{table2.png} 
\caption{Target gender x self-handicapping x gender of participant interaction for sympathy towards the target
Note: For Target Gender 0= female, 1= male, SelfHandicapping, 0= self-handicapping target, 1= non self-handicapping target, Gender (of participants) 1= male, 2= female.}
\label{table2}
\end{figure}



In addition, simple mean comparisons for target gender found that in self-handicapping trials, male participants liked (M= 4.76, SD= .26) and related to (M= 5.07, SD= .36) the target more and wanted the target as a friend (M=  4.80, SD= .29) and a roommate (M= 4.47, SD= .30) more if the target was female as opposed to male (liking: M= 3.92, SD= .29, relation: M= 3.69, SD= .40, friend: M= 3.64, SD= .33, roommate: M= 3.32, SD= .34). This unusual pattern is likely due to the small number of male participants in this study. In addition, simple mean comparisons for participant gender indicated that when women participants saw a female target who self-handicapped, they reported less liking (M= 3.82, SD= .18), sympathy (M= 3.61, SD= .24), and relation (M= 4.19, SD= .24) to her and wanted her less as a friend (M= 3.25, SD= .20), class partner (M= 1.76, SD= .22), and roommate (M= 3.04, SD= .21) than men participants did (liking: M= 4.76, SD= .26, sympathy: M= 4.51, SD= .36, relation: M= 5.07, SD= 36, friend: M= 4.80, SD= .29, class partner: M= 2.90, SD= .33, roommate: M= 4.47, SD= .30). Thought this result appears to be additional support that women were more sensitive to self-handicapping behavior, it should again be interpreted with caution since there were very few male participants. 

A novel, central component of this research was to test if the domain in which a target self-handicaps affects how self-handicapping behavior is perceived by an observer, in addition to the influence of gender of the target and gender of the observer. Results indicated several significant findings regarding domain. First, a 3-way interaction between domain, self-handicapping, and gender of participant affected predictions of how well participants thought the target would do if they had stayed home and studied, shown in Table 3, ($F(1, 218)= 3.61, p< .06$). Simple mean comparisons showed that women participants predicted that targets who self-handicapped before their English exam (a stereotypically female domain) would have done better had they studied (M= 4.93, SD= .13) than male participants thought they would do. Similarly, women participants predicted that targets who self-handicapped before their English exam, would have done better had they studied (M= 4.93, SD= .13) than target students who did not self-handicap  in this domain (M = 4.55, SD= .13). These results did not hold for women�s perceptions of the targets that had an upcoming a science exam. Men who saw a self-handicapping target predicted that Science students (stereotypically a �male� domain), but not English students, would do better if they had studied (M= 4.86, SD= .18) than if a non self-handicapping target was seen (M= 4.18, SD= .24). Additionally, men indicated that self-handicapping science students (M= 4.86, SD= .19) would have done better than English students (M= 4.26, SD= .23) if they had studied. Thus, self-handicapping views were more critical when participant gender and domain were matched in terms of gender stereotypes of the fields. 

\begin{figure}[h]
\centering
\includegraphics*[width=  .9 \columnwidth]{table3.png} 
\caption{Domain of Target x Self-Handicapping x Gender of Participant interaction for predictions of how well the target would do if they had stayed home and studied.
Note: For Domain 0= English, 1= science, SelfHandicapping 0= self-handicapping target, 1= non self-handicapping target, Gender (of participants) 1= male, 2= female. }
\label{table3}
\end{figure}


Second, there was a 2-way interaction between target gender and domain for ratings of how much participants would like to have the target as a friend, ($F(1, 215)= 5.19, p< .03$). Participants wanted a female target (M= 4.33, SD= .18) as a friend more than a male target (M= 3.75, SD= .18) in a science domain. They also wanted a female target more as a friend when she was in a science domain (M= 4.33, SD= .18) than an English domain (M= 3.79, SD= .18). These results indicate that people wanted to associate with women who were in science, more than men in science or women in English.

Third, results were analyzed in terms of participants� own division of major (if their division included English or Science) in relation to the domain of the target. The only significant effect was for English majors ratings for the future potential of targets within their domain. A 2 (Domain: english vs. science) x 2 (Participant�s division: English vs. not English) ANOVA revealed a main effect of domain, such that targets in English were viewed as having higher future potential in their domain than targets in science, ($F(1, 246)= 4.84, p< .03$). This is shown in Table 4.  This was qualified by a significant interaction between domain and participants� division of major, ($F(1, 246)= 5.82, p< .02$). Participants majoring in literature or languages (which includes English) rated targets in English (M= 3.60, SD= .22) as having higher future potential in their domain compared to participants majoring in any other division (M= 4.38, SD= .22). Thus, participants who associated with English viewed English targets more favorably in terms of their future potential in their domain.

\begin{figure}[h]
\centering
\includegraphics*[width=  .9 \columnwidth]{table4.png} 
\caption{Domain of Target x English Major Participants interaction for future potential of target in their domain.
Note: For Domain 0= English, 1= science, For Literature and Languages, 0= non-English major participants, 1= Literature and Language major participants}
\label{table4}
\end{figure}














\section{Discussion}
\subsection{Review of Results}
	There were four binary independent variables in this study (participant gender, target gender, domain, and whether or not the target self-handicapped), and a host of dependent variables as well, resulting in a large number of results.  The results lie in three main categories: replications, confirmations and new directions.  

The first category, \textit{replications}, refers to results which agreed with the prior literature on this and related topics.  The main article of interest, which also served as the springboard for this research was by Hirt, McCrea and Boris \cite{hirt2003}.  In their paper, the sex differences were entirely found in terms of the participant's gender, rather than the target's gender.  What they found was a polarization of the views of female participants such that self-handicappers were viewed more negatively and non-self-handicappers were viewed more positively than they were by male participants.  

This result was also found in our study, in that female participants felt (on several scales) more friendly toward non-self-handicappers.  Due to the small number of male participants, the contrasting data for male opinions was ambiguous.  

A main effect of self-handicapping was found, showing that participants were more favorable to those who did not self-handicap.  This result is also in line with prior research in the field.  What is odd is that in the present investigation as opposed to previous results, targets who did not self-handicap were rated as having higher future potential in their field, whereas those who did self-handicap were rated as likely to do better on the exam if they were to stay home and study.  These two results are usually coordinated, and usually both favor the self-handicapper, and this discrepancy with prior research bears further investigation.  

One potential reason for the discrepancy is the nature of the sample in the present study.  The college students who responded to various communications about an online survey were attendees at Reed College, a prestigious liberal arts college in Portland Oregon.  The academic culture of Reed is atypical of collegiate America in that hard work and dedication as well as natural ability are prized more highly than in normal college populations.  At this school, dedication to ones studies is met with respect, and procrastination with opprobrium.  This emphasis, and high regard for the ability to remain focused and to study for tests and attend to one's work is a possible reason for the inverse pattern of results found in this study as opposed to prior research.  

\textit{Confirmatory} results were those that informed us that our paradigm was indeed measuring what we aimed at, and was inducing the proper opinions in our participants.  

The main effect of self-handicapping described above indicates that (with the exception of the assessments of future potential), the paradigm successfully manipulated self-handicapping.  That said, less than 10\% each of male and female participants responded that targets might be self-handicapping to show off possible success, and less than 50\% each of male and female participants responded that targets might be self-handicapping as an excuse for failure.  These numbers indicate that although the self-handicapping binary did produce results, a different way of describing it in future research might more strongly induce an understanding of the situation in participants.  

The other manipulation in this study was of stereotypical views of academic domains.  Although there was no direct measurement of this manipulation, it will be seen that english and science did produce some gendered results.  These results show that the gender stereotypes in academia did indeed align along lines of domain.  

\textit{New Directions} were results which moved beyond prior research, and which were not sought as manipulation checks on our own work.  
	
There was further confirmation also of Hirt et. al.'s hypothesis which was qualified by an interaction with target gender.  For female participants, whether or not the target self-handicapped was extremely important as mentioned above, and this was true both for male and female targets.  Views were generally much more positive toward targets who did not self-handicap.  If the target was female however, these opinions were much stronger than if the target was male, and this constituted an overall interaction of participant and target genders.  These results are independent of domain (and of course, since Hirt et. al. did not examine different domains).  

Domain of study was one of the two novel variables in this study, and some clear results involving domain did emerge.  In particular, the three-way interaction between domain, self-handicapping and participant gender showed a fascinating pattern of results.  Female participants considered studying to be effective in English, but not in science whereas male participants considered studying to be effective in science, but not in English.  This pattern reveals that self-handicapping is effective at defending against others' judgements of one's ability only if those others are stereotypically associated with the domain in which the ability is being judged.  This topic certainly demands further study, as it may indicate limited usefulness of the ability-defending characteristics of self-handicapping.  It is likely however, that the negative social consequences (decreased likability) of self-handicapping will extend across populations and domains.  

One final result of interest is that regardless of the participants' are of study, they on average rated the english students as opposed to science students as having higher future potential.  The only break to this pattern was those participants in the division of literature and languages (including all english participants, for whom the trend was statistically stronger.  This data indicates again that the domain of study can create biases in the sample, and Hirt et. al. as well as other prior researchers should be cautioned for using only a history test in their manipulations.  Furthermore, this data suggests that participants' major in college (or at least their area of primary interest), can have additional biasing effects.  There may therefore, be many more subtle stereotyping or in-group identification effects biasing the data than were initially obvious.   

\subsection{Implications}
	The implications of this research for the field are largely cautionary.  Contrary to Hirt et. al. we found a several target gender effects, but perhaps more interestingly, they were primarily interaction effects which also involved the gender of the participant.  

	As a caution to others who might want to follow in the footsteps of Hirt et. al. (and others in the SH field), we found significant interactions involving the domain of study.  Some prior research has looked at only history, or has looked only at targets of a certain gender, and these choices could be capturing only part of the picture.  
	
	Stereotyping and self-handicapping are both well-researched domains, with long-established research procedures and large bodies of literature from which to draw.  The present research investigates the possibility that there may be interactions between these seemingly separate fields of inquiry.  The answer is unclear although both of these elements were certainly present in this study.  

The literature on self-handicapping is fraught with commentary about impression management and self-presentation.  These refer to conscious or non-conscious attempts to change others opinions about oneself by controlling the flow of information in a social situation.  

What the present study has shown is first that all previous conclusions regarding self-handicapping may be correct - that it is useful for impression management in terms of ability, but may have deficits with regard to likability.  Secondly however, the present study has also shown that these effects are dependent on other factors as well, such as the gender of the audience, and the domain of study.  For instance, a person who self-handicapped on a science exam would be well-defended (in terms of ability, not likability) from their male peers, but the attempted impression management would fail completely when directed at female peers.  Likewise a student self-handicapping on an english assessment would only be able to manage the impressions of female peers, rather than male.  This is perhaps the most striking result from the current investigation.  


\subsection{Potential Problems}
Two potential problems with this experimental design, subject pool and recruiting methods are evident.  Firstly the similarity between vignettes was sufficient to change participants' answers.  That is, there were statistical effects of vignette placement which led to the second set of responses from each participant being eliminated from the analysis.  Secondly the small number of male respondents resulted in a very small number of statistically significant results which involved looking only at this sample.  This is frustrating because many interesting statistics were calculated with the set of female participants, and having a male participant pool to compare them to would make a much more meaningful comment.  It should be noted however, that neither of these inversely affected the results reported here, but instead, these issues served to make the pattern of secure results more inconsistent.  


\subsection{Future Directions}
In terms of future directions, the results of this investigation have seemingly asked more questions than they have answered.  One interesting point is that while self-handicappers were expected to perform better on the exam if they were to study, non-self-handicappers were rated as having more future potential in the domain.  This is interesting because in terms of raw scores, the self-handicappers are expected to out-perform the non-self-handicappers, but to fall behind them in the field.  Thus participants have some sense of how hard work and dedication, rather than ability, can turn into success.  This could be an interesting direction for future study.  

Another direction for future research is the domain of study of the participants.  In a large enough sample size, it would not be surprising to see participant domain x participant gender x target domain interactions emerging, in the case, say, of female physics student participants holding unique views about self-handicapping in science.  

This study did challenge the conclusion of Hirt et. al. that gender of target is irrelevant for self-handicapping concerns, although no clear picture has yet emerged showing what the effects of this variable are.  

Finally, if participants' genders or area of study is enough to bias their views of self-handicappers, in some way (sometimes those self-handicappers will need to have extra characteristics such as being male or female), a very relevant study could easily be performed on educators impressions of self-handicappers to discover whether this pattern of results extends to supposedly impartial authority figures as well.  

\section{Concluding Remarks}
We hypothesized that men and women self-handicapping in stereotypically male or female domains would produce different responses from participants.  While this hypothesis was not fully supported, (statistical significance was not reached on this measure, many other interesting results were discovered indicating that not only was our paradigm successful in replicating previous findings and in manipulating the intended ideas, but that domain of study, gender of target and gender of participant are all important factors when considering impression management through self-handicapping.  Much research remains to be done at this intersection of several sub-fields of social psychology.  


\bibliography{shrp}                                                                                                                                                

\clearpage
\onecolumn
\appendix
\section{Appendix A: Vignettes}
Instructions when participants encountered these vignettes read as follows: ``You will read two short vignettes, each followed by several questions. Click next to continue to the first vignette and set of questions."

\

FEMALE/ENGLISH/SELF HANDICAPPING:
Kristin has spent 45 minutes studying unproductively for her English midterm, which is the following morning. Her friend then calls her up, inviting her to go to a movie for the evening. Kristin decides to stop studying for her English midterm in order to join her friend at the movie. She does not get home until it is very late, at which time she goes to bed without any further preparation for her exam. She later learns that her grade on the midterm is a C.

\

MALE/ENGLISH/SELF HANDICAPPING:
Michael has spent 45 minutes studying unproductively for his English midterm, which is the following morning. His friend then calls him up, inviting him to go to a movie for the evening. Michael decides to stop studying for his English midterm in order to join his friend at the movie. He does not get home until it is very late, at which time he goes to bed without any further preparation for his exam. He later learns that his grade on the midterm is a C.

\

FEMALE/SCIENCE/SELF HANDICAPPING:
Kristin has spent 45 minutes studying unproductively for her science midterm, which is the following morning. Her friend then calls her up, inviting her to go to a movie for the evening. Kristin decides to stop studying for her science midterm in order to join her friend at the movie. She does not get home until it is very late, at which time she goes to bed without any further preparation for her exam. She later learns that her grade on the midterm is a C.

\

MALE/SCIENCE/SELF HANDICAPPING:
Michael has spent 45 minutes studying unproductively for his science midterm, which is the following morning. His friend then calls him up, inviting him to go to a movie for the evening. Michael decides to stop studying for his science midterm in order to join his friend at the movie. He does not get home until it is very late, at which time he goes to bed without any further preparation for his exam. He later learns that his grade on the midterm is a C.

\

FEMALE/ENGLISH/NON-SELF HANDICAPPING
Kristin is studying for her English midterm, which is the following morning. Her friend then calls her up, inviting her over to her house for dinner. Kristin decides to spend the evening at her apartments, so that she can study for her English midterm. She later learns that her grade on the midterm is a C.

\

MALE/ENGLISH/NON-SELF HANDICAPPING
Michael is studying for his English midterm, which is the following morning. His friend then calls him up, inviting him over to his house for dinner. Michael decides to spend the evening at his apartments, so that he can study for his English midterm. He later learns that his grade on the midterm is a C.

\

FEMALE/SCIENCE/NON-SELF HANDICAPPING
Kristin is studying for her science midterm, which is the following morning. Her friend then calls her up, inviting her over to her house for dinner. Kristin decides to spend the evening at her apartments, so that she can study for her science midterm. She later learns that her grade on the midterm is a C.

\

MALE/SCIENCE/NON-SELF HANDICAPPING
Michael is studying for his science midterm, which is the following morning. His friend then calls him up, inviting him over to his house for dinner. Michael decides to spend the evening at his apartments, so that he can study for his science midterm. He later learns that his grade on the midterm is a C.
\clearpage
\section{Appendix B: Questionnaires}

     Performance and Ability Questions:
     
Directions: �Please answer the following questions using the scale provided.�
     
0-6 scale: 0= Poor, 3= Average, 6= Excellence
     
     \
     
1.	How well do you predict [name] would do if he/she stayed home and studied the night before?
     
2.	Assess [name�s] future potential in [domain] 
     
     \
     
Interpersonal Questions: 

Directions: �Please answer the following questions using the scale provided.�

0-6 scale: 0= Poor, 3= Average, 6= Excellence

\

1.	How much do you like [insert name from vignette] 

2.	How sympathetic do you feel towards [name] 

3.	How well do you relate to [name]

4.	How similar do you feel towards [name]

5.	How much would you like to have [name] as a friend

6.	How much would you like to have [name] as a partner on class project

7.	How much would you like to have [name] as a roommate 

\

Perceived Motivation Questions (Handicapping Vignettes): 

Directions: �Here is a list of possible motives for [name�] behavior. Please mark all motives that you believe are applicable.�

1.	[he/she] is unmotivated

2.	[he/she] lacks discipline

3.	[he/she] appears pressured

4.	[he/she] felt prepared

5.	[he/she] needed a study break

6.	[he/she] wants to have an excuse for failure*

*indicator of self-handicapping

7.	[he/she] wanted to show off possible success*

*indicator of self-handicapping

Perceived Motivation Questions (Non-Handicapping Vignettes): 

\

Directions: �Here is a list of possible motives for [name�] behavior. Please mark all motives that you believe are applicable.�

1.	[he/she] is motivated

2.	[he/she] is disciplined

3.	[his/her] peers value hard work

4.	[he/she] does not feel prepared

5.	[he/she] has not trouble staying focused

6.	[he/she] is determined to succeed

7.	[he/she] wanted to show off possible success
\end{document}




