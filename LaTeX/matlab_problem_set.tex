\documentclass[aps,pre,nofootinbib]{revtex4}

\usepackage{amsmath,amssymb,amsfonts,amsthm}
\usepackage{graphicx}
\usepackage{bbm}
\usepackage{pdfsync}
\usepackage{color}

\begin{document}

\title{MatLab Problem Set 1}

\author{Gray Davidson}
\affiliation{Department of Physics, Reed College, Portland, Oregon,  97202, USA}
\affiliation{For Physics 331, week 8}
\date{\today}

\maketitle



I do not envy you the responsibility of reading this document.  I did not do any of the extra problems since the 18 that I did do already took about double the three hour lab time.  


\begin{enumerate}

\item 
% This LaTeX was auto-generated from an M-file by MATLAB.
% To make changes, update the M-file and republish this document.



\sloppy
\definecolor{lightgray}{gray}{0.5}
\setlength{\parindent}{0pt}

 I played with variable assignment and with commands for a bit, then created an m-file, and published it in \LaTeX format.  I also watched the video, which was helpful in publishing.

   \begin{verbatim}
diary davidson_lab_XIII

x = [1 3 0 -1 5]

H_P_Lovecraft = 666

y = sin (x)

z = x/y

load earth
image (X) ; colormap (map)
axis image

why
why (2)
why (3)
why (2)
why (4)
why (16)

diary off
\end{verbatim}

       \color{lightgray} \begin{verbatim}
x =
    1     3     0    -1     5

H_P_Lovecraft =
  666

y =
   0.8415    0.1411         0   -0.8415   -0.9589

z =

  -1.1412

The rich rich and tall and good system manager suggested it.
You insisted on it.
Jack obeyed a bald engineer.
You insisted on it.
Some not excessively young hamster insisted on it.
It should be obvious.
\end{verbatim} \color{black}

\item An example of MATLAB code that will produce this error is the command 

>> sin (saturday)

if the variable ``saturday" has not previously been assigned a value.  

\item    \begin{verbatim}
A = [16 3 2 13; 5 10 11 8; 9 6 7 12; 4 15 14 1]
B = A'
sum (A (1:4))
sum (A (5:8))
sum (A (9:12))
sum (A (13:16))
sum (B (1:4))
sum (B (5:8))
sum (B (9:12))
sum (B (13:16))
\end{verbatim}

       \color{lightgray} \begin{verbatim}
A =

   16     3     2    13
    5    10    11     8
    9     6     7    12
    4    15    14     1


B =

   16     5     9     4
    3    10     6    15
    2    11     7    14
   13     8    12     1


ans =

   34


ans =

   34


ans =

   34


ans =

   34


ans =

   34


ans =

   34


ans =

   34


ans =

   34

\end{verbatim} \color{black}


\item The magic square appears in the upper right hand corner, done in relief on the side of the building, between the angel's wings and the hourglass.  

\item 
   \begin{verbatim}
Eisenhower = 1944
data_set = [5 5 7 6 7 4]
H_P_Lovecraft = 666
x = rand (1,10)
y = sin (x)
z = y/x
who
\end{verbatim}

       \color{lightgray} \begin{verbatim}
Eisenhower =

       1944


data_set =

    5     5     7     6     7     4


H_P_Lovecraft =

  666


x =

   0.7060    0.0318    0.2769    0.0462    0.0971    0.8235    0.6948    0.3171    0.9502    0.0344


y =

   0.6488    0.0318    0.2734    0.0462    0.0970    0.7335    0.6403    0.3118    0.8135    0.0344


z =

   0.8965


Your variables are:

Eisenhower     data_set       y
H_P_Lovecraft  x              z

\end{verbatim} \color{black}

\item Indeed I can.

\item     \begin{verbatim}
h_bar = 6.626068*10^-34 %m^2 kg / s         Planck's Constant
k = 1.3806503*10^-23 %m^2 kg / s^2K         Boltzmann Constant
avogadro = 6.0221415*10^23 %                Avogadro's Number
e = 1.60217646*10^-19 %coulombs             Elementary Charge
m_e = 9.10938188*10^-31 %kilograms          Electron Mass
\end{verbatim}

       \color{lightgray} \begin{verbatim}
h_bar =

  6.6261e-34


k =

  1.3807e-23


avogadro =

  6.0221e+23


e =

  1.6022e-19


m_e =

  9.1094e-31

\end{verbatim} \color{black}

\item
    \begin{verbatim}
a = [1 8 3 -2 5 -13 -2 0 4]
size (a)
disp (a)
b = a'
size (b)
\end{verbatim}

        \color{lightgray} \begin{verbatim}
a =

     1     8     3    -2     5   -13    -2     0     4


ans =

     1     9

     1     8     3    -2     5   -13    -2     0     4


b =

     1
     8
     3
    -2
     5
   -13
    -2
     0
     4


ans =

     9     1

\end{verbatim} \color{black}

\item
    \begin{verbatim}
x = 1:13
y = 0:pi/2:4*pi
\end{verbatim}

        \color{lightgray} \begin{verbatim}
x =

     1     2     3     4     5     6     7     8     9    10    11    12    13

y =

         0    1.5708    3.1416    4.7124    6.2832    7.8540    9.4248   10.9956   12.5664

\end{verbatim} \color{black}

\item

    \begin{verbatim}
A = [16 3 2 13; 5 10 11 8; 9 6 7 12; 4 15 14 1]
A (3,3)
A (1:4)
A ([1 6 11 16])
A (1:4:16)
\end{verbatim}

        \color{lightgray} \begin{verbatim}
A =

    16     3     2    13
     5    10    11     8
     9     6     7    12
     4    15    14     1

ans =

     7

ans =

    16     5     9     4

ans =

    16    10     7     1

ans =

    16     3     2    13

\end{verbatim} \color{black}

\item

    \begin{verbatim}
r = 1:4
r.^2
r*r'
s = [1 2;3 4]
s^2
s.^2
\end{verbatim}

        \color{lightgray} \begin{verbatim}
r =

     1     2     3     4

ans =

     1     4     9    16

ans =

    30

s =

     1     2
     3     4

ans =

     7    10
    15    22


ans =

     1     4
     9    16

\end{verbatim} \color{black}

\item Exercises on Hahn's p. 38
\begin{enumerate}
\item
\begin{enumerate}
\item Prediction: 7

\item Prediction: 4

\item Prediction: 3/2

\item Prediction: 16

\item Prediction: 8/3

\item Prediction: (e-2)/2 which is ~.35

\end{enumerate}

    \begin{verbatim}
1+2*3
4/2*2
1+2/4
2*2^3
2^(1+2)/3
1/2*2.71828183-1

\end{verbatim}

        \color{lightgray} \begin{verbatim}

ans =

     7

ans =

     4

ans =

    1.5000


ans =

    16

ans =

    2.6667


ans =

    0.3591
\end{verbatim} \color{black}


\item
\begin{enumerate}
\item

    \begin{verbatim}

1/(2*3)

ans =

    0.1667
\end{verbatim}

\item
 \begin{verbatim}

2^(2*3)


ans =

    64

\end{verbatim}
\end{enumerate}
\end{enumerate}
\item
    \begin{verbatim}
[5, 6, 7] .^ [1, 2, 3]
\end{verbatim}

        \color{lightgray} \begin{verbatim}
ans =

     5    36   343

\end{verbatim} \color{black}

This result is explained by understanding of array operations.  The (.) in front of the \^ symbol, which denotes exponentiation, makes that operation an array operation, and therefore cases it to proceed component-wise.  thus: $5^1$ is 5, $6^2$ is 36 and $7^3$ is 343.

\item
\begin{enumerate}
\item
   \begin{verbatim}
1.+[2 3 -1]

ans =

     3     4     0
     \end{verbatim}
     \item
        \begin{verbatim}
3.*[1 4 8]

ans =

     3    12    24
     \end{verbatim}
     \item
        \begin{verbatim}
[1 2 3].*[0 -1 1]

ans =

     0    -2     3
     \end{verbatim}
     \item
        \begin{verbatim}
[2 3 1].^2

ans =

     4     9     1
     \end{verbatim}
\end{enumerate}

\item

Instead of this command using `for,'
   \begin{verbatim}
for c = 0:12
B = sin (pi*c/6);
disp ( [c/6*pi B] )
end
\end{verbatim}

        \color{lightgray} \begin{verbatim}    
    0               0
    0.5236    0.5000
    1.0472    0.8660
    1.5708    1.0000
    2.0944    0.8660
    2.6180    0.5000
    3.1416    0.0000
    3.6652   -0.5000
    4.1888   -0.8660
    4.7124   -1.0000
    5.2360   -0.8660
    5.7596   -0.5000
    6.2832   -0.0000
\end{verbatim} \color{black}

We use this command, using a vector.  The result is a faster calculation which is irrelevant with only 12 inputs, but which becomes extremely important when one is dealing with thousands of inputs.
   \begin{verbatim}

A = [0:1/6:2]';
B = sin (pi*A);
disp ( [A*pi B] )
\end{verbatim}

        \color{lightgray} \begin{verbatim}     0     0

         0         0
    0.5236    0.5000
    1.0472    0.8660
    1.5708    1.0000
    2.0944    0.8660
    2.6180    0.5000
    3.1416    0.0000
    3.6652   -0.5000
    4.1888   -0.8660
    4.7124   -1.0000
    5.2360   -0.8660
    5.7596   -0.5000
    6.2832   -0.0000

\end{verbatim} \color{black}

\item
\begin{enumerate}
\item
\begin{enumerate}
\item
 \begin{verbatim}
2^2*3/4+3
ans =

     6
\end{verbatim}
\item
 \begin{verbatim}
2^(2*3)/(4+3)

 ans =

    9.1429

    \end{verbatim}
\item
\begin{verbatim}
-4^2
ans =

   -16

\end{verbatim}
\end{enumerate}
\item
\begin{enumerate}
\item
\begin{verbatim}
1/(2*pi^(1/2))


ans =

    0.2821
    \end{verbatim}
\item
\begin{verbatim}
(1-2/(3+2))/(1+2/(3-2))


ans =

    0.2000

\end{verbatim}
\item
\begin{verbatim}
(1.23e-5+5.678e-3)*4.567e-5


ans =

   2.5988e-07

\end{verbatim}
\end{enumerate}
\item
\begin{verbatim}


(2*3+4)/(5*(6+1))^2
although I can't see why.

ans =

    0.0082

\end{verbatim}
\item
\begin{verbatim}


n = 1:5
m = (1./n).^2

n =

     1     2     3     4     5


m =

    1.0000    0.2500    0.1111    0.0625    0.0400
\end{verbatim}
\item
\begin{verbatim}
a = [2 -1 5 0]
b = [3 2 -1 4]
prediction: [12 -1 49/5 0]
c = 2*a+a.^b

a =

     2    -1     5     0


b =

     3     2    -1     4


c =

   12.0000   -1.0000   10.2000         0
\end{verbatim}
\end{enumerate}


\item

    \begin{verbatim}
disp ( 'Pilate said, ''What is truth?''' );
disp ( ['Douglas Adams answered, ', '''My good Sir, surely truth is ', num2str(42),'.'''] );
\end{verbatim}

        \color{lightgray} \begin{verbatim}Pilate said, `What is truth?'
Douglas Adams answered, `My good Sir, surely truth is 42.'
\end{verbatim} \color{black}

\item

    \begin{verbatim}
for C = 1:5, disp (C), end
\end{verbatim}

        \color{lightgray} \begin{verbatim}
     1
     2
     3
     4
     5
\end{verbatim} \color{black}
    \begin{verbatim}
for C = 1:3, disp (C), end
\end{verbatim}

        \color{lightgray} \begin{verbatim}

     1
     2
     3

\end{verbatim} \color{black}

\end{enumerate}

\end{document}